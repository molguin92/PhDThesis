\section{Summary of the key contributions of this thesis}\label{sec:summary_contributions}
\todo[inline]{Needs tweaking to match scope and related work.}
\todo[inline]{Missing discussion, presentation, of alternative approaches explicitly.}
\todo[inline]{Challenges and how we tackled them.}

\subsection{Primary contributions}

\subsection{Secondary contributions}


This thesis presents three core contributions to the existing body of research in edge computing, as well as a number of secondary ones.
Firstly, we introduce a methodological approach to studying system responsiveness versus resource consumption trade-offs in edge-bound latency-sensitive applications such as \gls{WCA} and \gls{NCS}.
This approach is based on the emulation of the client-side workload, while maintaining the \emph{real} server-side process as well as network stack.
We validate this methodology by example, presenting case studies of its application in the context of \gls{WCA} and \glspl{NCS}, and introduce a software framework for the orchestration of the edge computing testbeds necessary for its implementation.

Secondly, we present a deeper exploration of this methodology for \gls{WCA} and introduce the first ever model for the end-to-end emulation of human timing behavior in \gls{WCA}.
This model is built from a thorough characterization of these behaviors, the data for which we obtain from a comprehensive human-subject study.

Finally, as an extension and application of our first and second contributions, we explore the implications of our methodology and human user model on the optimization potential of \gls{WCA} deployments.
In particular, we study the sampling and energy consumption behaviors of these systems with and without considering realistic human behavior.
We conclude that significant improvements can be achieved through the use of our human user model.

