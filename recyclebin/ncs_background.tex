\glspl{NCS}~\cite{Gupta2010NCSOverview}, a type of \gls{CPS} wherein multiple networked actuators and sensors form a part of the same automatic control system will benefit from the adoption of these technologies.
Depending on the physical system being controlled, \glspl{NCS} can have stringent timing and reliability requirements for communication that conventional cloud paradigms and cellular networks cannot meet~\cite{Wan2020Efficient}.
This necessitates sophisticated tools for the performance evaluation of future system architectures, as well as novel NCS design paradigm.

Due to their potential advantages for industrial and commercial settings, there exist works~\cite{Zhang2016Survey} dedicated to the modelling and performance characterization of \glspl{NCS}, improving NCSs by distributing control functions across networks, facilitating centralized coordination, control, and monitoring.

One the one hand, related literature in \glspl{NCS} leverages to a large extent theoretical models, at the price of being able to capture networked systems effects only on a coarse level.
%follows a theoretical approach, and only a small fraction of it deals with experimental studies.
On the other, there exist several approaches when considering experimental methodologies.
%A number of works concerning \glspl{NCS} deal with experimental studies.
%\glspl{NCS} have an inherently inter-domain nature intertwining knowledge from the fields of communications, computing, and control theory in ways that cannot be studied in isolation, leading to various different approaches to such studies.
One such approach uses setups in which the complete system is built on top of real hardware.
This approach is employed in the works of Baumann \emph{et al.}~\cite{Baumann2018LowPower} and Cuenca \emph{et al.}~\cite{Cuenca2019UAV}; in both of these, the authors implement their approach on physical testbeds.
Conversely, other studies choose to instead use completely \emph{simulated} \gls{NCS} setups.
The authors in\ \cite{Ma2019DynamicSched} have opted for such an approach.
These studies often employ combinations of physical and network simulation tools trying to capture the complex dynamics of \glspl{NCS}.
Finally, some experimental studies instead employ \emph{virtualized} approaches, in which either
\begin{inlineenum}[itemjoin={{; }}, itemjoin*={{; or }}]
    \item a real network interacts with a simulated or emulated control system~\cite{Wang2020VoltageControl}
    \item a simulated network interacts with a real control system~\cite{Natale2004InvPendEthernet}.
\end{inlineenum}

As evidenced above, experimental research in \glspl{NCS} includes varied heterogeneous hardware and software platforms, methodologies and key performance indicators.
This, in turn, leads to hardware, software, and methodology fragmentation, as different studies tend to prefer approaches more favored in their respective communities.
Furthermore, existing studies tend to focus on individual aspects and components of a system, thus producing results which do not provide a complete image of the \gls{NCS}.
This has caused a gap in knowledge pertaining to the reproducibility and comparison of experimental studies on these systems.

Zoppi \emph{et al.}~\cite{Zoppi2020NCSBench} made the first (and to the best of our knowledge, the only) attempt at tackling this challenge in their work.
In their work, they proposed a platform called NCSBench, to be used for reproducible benchmarking in NCS.\
Their methodology utilizes joint knowledge of control, computation, and communication.
In their work various architectural elements and the corresponding delays associated with the NCS are modelled.
Multiple experimental parameters and certain observable key performance indicators are defined and utilized in the implementation.
This work however utilizes a physical LEGO\textregistered{}\ Mindstorms EV3 Core Set\texttrademark{}\  based plant for the implementation, preventing instantaneous changes in plant characteristics and component parametrizations.
Furthermore, relying on physical objects like an inverted pendulum limits scalability of the experimentation in practice.
