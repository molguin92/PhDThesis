\phantomsection\addcontentsline{toc}{chapter}{Abstract}
\begin{abstract}
The rise of Edge Computing has enabled a wide range of latency-sensitive applications.
In particular, the combination of Edge Computing and 5G offers potential for massive deployments of latency-sensitive applications such as \glspl{CPS} and \gls{MAR}.
However, multiple challenges remain in scaling and understanding these systems, including the optimization of latency through multiple processing steps and trade-offs in wireless system choice, protocols, hardware, and algorithms.
In the case of \gls{MAR}, sensory inputs need to be pre-processed and compressed, and the backend algorithms designed for efficient processing and minimal latency.
For \glspl{CPS}, reliable data transmission is critical to prevent system failure, and issues such as congestion and fluctuations in the wireless channel can lead to higher end-to-end latency.
There is a growing need for methodological approaches that can accurately evaluate the performance of these applications and provide insights into their design and optimization for Edge Computing.

This dissertation attempts to address this challenge by proposing three primary contributions to Edge Computing research.
Our first and main contribution corresponds to a methodological approach to studying the trade-offs between system responsiveness and resource consumption in edge-bound latency-sensitive applications.
The approach involves the emulation of the client-side workload while maintaining the real server-side process and network stack, in order to maintain the realism of network effects on the system.
This methodology is validated through case studies in the context of \gls{WCA} and \gls{NCS}, and we introduce a software framework for the orchestration of Edge Computing testbeds for the automated deployment of our methodology. 

Our second and third key contributions correspond to a deeper exploration of this methodology for \gls{WCA} and the exploration of its implications on the optimization potential of these applications.
We present the first ever model for end-to-end emulation of human timing behavior in \gls{WCA} and explore optimization approaches which leverage this model.
From our exploration, we conclude that significant improvements can be achieved by considering realistic human behavior.
\end{abstract}

\vspace{3em}

\setlength{\leftskip}{0.3 cm} \textbf {Keywords:} Lorem, Ipsum, Dolor, Sit, Amet

%%%%%%%%%%%%%%%%%%%%%%%%%%%%%%%%%%%%%%%%
%%%%%%%% SWEDISH ABSTRACT %%%%%%%%%%%%%%
%%%%%%%%%%%%%%%%%%%%%%%%%%%%%%%%%%%%%%%%
\newpage
\selectlanguage{swedish}
\phantomsection\addcontentsline{toc}{section}{Abstract (Swedish)}
\begin{abstract}
\noindent \lipsum[1]
\end{abstract}
\selectlanguage{english}