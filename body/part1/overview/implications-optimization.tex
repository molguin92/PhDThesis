\section{Implications for the optimization of \ac{WCA}}

\begin{enumerate}
    \item\label{item:contrib:footprint} Using the model described above, we study the implications of realistic human behavior on the application lifetime footprint of \ac{WCA}.
    In concordance with previous work~\cite{olguinmunoz:impact2021}, we find dependencies between system responsiveness and human step execution times that lead to substantially different application lifetimes when compared to a first-order baseline which does not take into account human behavior.
    \item\label{item:contrib:optimization} Finally, we study the potential for optimization in \ac{WCA} when considering human behavior using our model.
    We develop a generic model for the stochastic optimization of resource consumption versus responsiveness trade-offs in these applications, which we apply to two potential avenues for \ac{WCA} optimization; number of processed samples and energy consumption per step.
    Compared to the state-of-the-art, our optimization model results in up to \textasciitilde\SI{60}{\percent} fewer samples processed or an average of \SI{20}{\percent} less energy consumed per step, while maintaining comparable levels of responsiveness.
\end{enumerate}

Next, we have explored the impact of such a realistic model on the application lifetime footprint of \ac{WCA} applications.
We have shown that less realistic modeling approaches which do not take into account higher-order effects on execution time distributions can potentially lead to significant misestimations of application footprint.

Finally, we have delved into the potential for optimization in \ac{WCA} systems using the previously discussed timing models.
We have proposed a novel stochastic optimization framework for resource consumption-system responsiveness trade-offs in \ac{WCA} which results in an adaptive sampling strategy.
We have shown that this framework is applicable to a myriad of metrics in these applications, and showcased experimental results employing this framework for the minimization of number of samples processed per step and total energy consumption per step.
Our results show up to a \SI{50}{\percent} increase in performance with respect to state of the art when optimizing for number of samples, and up to a \SI{30}{\percent} improvement when optimizing for energy consumption, proving thus the value of such frameworks for the design of \ac{WCA} applications.