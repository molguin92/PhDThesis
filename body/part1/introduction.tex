\chapter{Introduction}\label{chap:introduction}

\todo[inline]{
    Introduction here. Why do we need this work?
}

\todo[inline]{
    Reorganize EdgeDroid, Cleave, Ainur, together.

    Pull in multi-loop results for EdgeDroid?

    Think about experiments.
}

\section{Summary of the key contributions of this thesis}

This thesis presents three core contributions to the existing body of research in edge computing, as well as a number of secondary ones.
Firstly, we introduce a methodological approach to studying system responsiveness versus resource consumption trade-offs in edge-bound feedback-loop systems such as \ac{WCA} and \ac{NCS}.
This approach is based on the emulation of the client-side workload, while maintaining the \emph{real} server-side process as well as network stack.
We validate this methodology by example, presenting implementations of its application in the context of \ac{WCA} and \acp{NCS}, and introduce a software framework for the orchestration of the edge computing testbeds necessary for its implementation. 

Secondly, we present a deeper exploration of this methodology for \ac{WCA} and introduce the first ever model for the end-to-end emulation of human timing behavior in \ac{WCA}.
This model is built from a thorough characterization of these behaviors, the data for which we obtain from a comprehensive human-subject study.

Finally, as an extension and application of our first and second contributions, we explore the implications of our methodology and human user model on the optimization potential of \ac{WCA} deployments.
In particular, we study the sampling and energy consumption behaviors of these systems with and without considering realistic human behavior.
We conclude that significant improvements can be achieved through the use of our human user model.

\subsection{Overview of each included paper}

\cref{paper:olguinmunoz2018demoscaling,paper:olguinmunoz2019edgedroid,paper:olguinmunoz2022cleave,paper:olguinmunoz2022ainur} introduce and discuss this methodology and the necessary tools for its implementation.
\cref{paper:olguinmunoz2018demoscaling} presents a short, high level overview of this methodological approach as applied to \ac{WCA}, making very straightforward assumptions about human behavior in \ac{WCA} applications.
\cref{paper:olguinmunoz2019edgedroid} extends the discussion from \cref{paper:olguinmunoz2018demoscaling}, providing more details on the implementation of the necessary measurement framework and the tooling employed.
This work presents the first empirical results obtained with our methodological approach, providing thus an initial validation of its utility for research.
\cref{paper:olguinmunoz2019edgedroid} also discusses in detail the assumptions that were made about human behavior and reactions.
We assume in these works a human which is impervious to poor system performance, and suffers no annoyance, fatigue, frustration, nausea or other shortcomings of real human users.
The result is a model of a user who responds in a precisely reproducible and deterministic manner to the same system stimulus every time.

\cref{paper:olguinmunoz2022cleave}, on the other hand, presents a discussion on the implementation of this methodology for \acp{NCS}.
As discussed above, these applications are similarly difficult as \ac{WCA} to benchmark (in particular at scale on multi-tenant systems) due to their client-side complexity, but in particular because of their often extreme sensitivity to latency.
\todo[inline]{More argumentation or refer to previous section.}
We apply thus in this work our methodology to \ac{NCS} through the implementation of a tool for the emulation and subsequent deployment of these systems on edge computing infrastructure.
This tool allows us to both emulate the physical components of a relatively simple control system plant and deploy real algorithms for its control.
The software acts as a middleware which abstracts away the network from the development of these workloads, allowing for quick prototyping and deployment.
This work also showcases the scalability and flexibility of our approach, allowing us to deploy scenarios with a large number of loops without the need for domain-specific hardware.

\cref{paper:olguinmunoz2022ainur} presents a tangential contribution.
It introduces the software framework used in \cref{paper:olguinmunoz2023realistic,paper:olguinmunoz2022cleave} for the orchestration of the edge computing testbeds on which the developed tools were deployed.
This framework represents an ancillary contribution, crucial for the research presented in these works.
Without it, the experimental approach described in these works would not have been feasible.

Next, in \cref{paper:olguinmunoz2021impact,paper:olguinmunoz2023realistic} we delve deep into the implementation of our methodology for \ac{WCA}, as well as study its implications for optimization of these systems.
Although the above approach to human behavior in \ac{WCA} discussed represents a useful initial approximation, it is nonetheless not a realistic model of it.
In \cref{paper:olguinmunoz2021impact} we therefore take the first major step towards such a model, presenting a deep characterization of human behavior in these applications.
We develop this characterization through a human subject study with a cohort of \num{40} participants who were asked to interact with an instrumented \ac{WCA} application.
System responsiveness is altered in real-time during each execution of the task, and we record participants' reactions by measuring task- and system-related metrics, as well as biometrics from sensors placed on their bodies.
Participants were also asked to fill out two personality indicator questionnaires, allowing us to later correlate individual personality traits and measured reaction to changes in system responsiveness.

\cref{paper:olguinmunoz2023realistic} then concludes this line of work, building upon the insights and data obtained in \cref{paper:olguinmunoz2021impact} to develop the first ever data-driven model of human timings for \ac{WCA}.
The model is validated against previously obtained results, both through simulated, controlled executions and deployments on a real edge computing testbed.
This work also explores potential implications of this model for \ac{WCA} system optimization potential, particularly in the domains of energy consumption and sampling strategies.

\section{Structure of this thesis}

\todo[inline]{This thesis is structured as follows.}