\section{Introduction}

The advancement of technology has brought about new computing paradigms and network technologies that have the potential to revolutionize the way we think about and approach computing and communication.
One of the most promising of these is Edge Computing, which seeks to bring computing closer to the end user, reducing latency and increasing reliability.
Another key player in this space is 5G, the latest generation of mobile communication networks that offers higher data rates, lower latency, and improved reliability.
The combination of these two technologies holds great promise for the deployment of latency-sensitive applications, including \glspl{CPS}, \glspl{NCS}, and \gls{WCA} applications, among others.

Despite these advancements, there are still significant challenges in understanding and scaling these systems, particularly in regard to their cyber-physical nature, complexity, and requirements for low latency and high reliability.
The aim of this thesis is to make a contribution to the field by investigating the applications of a methodology for the study of latency-sensitive applications deployed on Edge infrastructure and network technologies such as 5G.
The proposed methodology aims to enhance the accuracy and realism of results related to Edge infrastructure and 5G networks, particularly in regard to network performance.
Our results will contribute to the development of new techniques and approaches for improving the performance and reliability of \gls{MEC}.

Our methodology is based on the emulation of target workloads on actual Edge infrastructure.
We replace the client side of the system with a realistic emulation of the desired behaviors, implemented in software deployed on \gls{COTS} general-purpose computing devices.
In our initial implementation, these correspond to low-cost, easily replaceable, and scalable Raspberry Pi 4 Model B \glspl{SBC}.

Emulating the workload component reduces complexity by moving it into the software domain, allowing for easier scaling through the use of cheap, COTS general-purpose hardware such as \glspl{SBC}.
It also preserves the realism of effects stemming from the hardware and network.
In particular, the methodology allows us to capture effects due to network factors such as contention, congestion control, and medium access, which are often of stochastic or chaotic natures and complex to capture in simulations.

The methodology also provides improved repeatability and replicability.
Repeating a study becomes a matter of re-running the workload on the same testbed, and studies can be replicated simply by obtaining the same or equivalent software workload and deploying it on a comparable testbed.
These are complex tasks to accomplish in real-world approaches, particularly when dealing with humans.

Our methodology provides a comprehensive and realistic assessment of the performance of latency-sensitive applications deployed on Edge infrastructure.
The approach provides valuable insights for researchers, system designers, and application developers, and will contribute to the development of new techniques and approaches for improving the performance and reliability of Edge Computing and 5G systems.

\section{Summary of the key contributions of this thesis}\label{sec:summary_contributions}

This thesis presents three core contributions to the existing body of research in edge computing, as well as a number of secondary ones.
Firstly, we introduce a methodological approach to studying system responsiveness versus resource consumption trade-offs in edge-bound latency-sensitive applications such as \gls{WCA} and \gls{NCS}.
This approach is based on the emulation of the client-side workload, while maintaining the \emph{real} server-side process as well as network stack.
We validate this methodology by example, presenting case studies of its application in the context of \gls{WCA} and \glspl{NCS}, and introduce a software framework for the orchestration of the edge computing testbeds necessary for its implementation.

Secondly, we present a deeper exploration of this methodology for \gls{WCA} and introduce the first ever model for the end-to-end emulation of human timing behavior in \gls{WCA}.
This model is built from a thorough characterization of these behaviors, the data for which we obtain from a comprehensive human-subject study.

Finally, as an extension and application of our first and second contributions, we explore the implications of our methodology and human user model on the optimization potential of \gls{WCA} deployments.
In particular, we study the sampling and energy consumption behaviors of these systems with and without considering realistic human behavior.
We conclude that significant improvements can be achieved through the use of our human user model.

\subsection{Overview of each included paper}

\cref{paper:olguinmunoz2018demoscaling,paper:olguinmunoz2019edgedroid,paper:olguinmunoz2022cleave,paper:olguinmunoz2022ainur} introduce and discuss this methodology and the necessary tools for its implementation.
\cref{paper:olguinmunoz2018demoscaling} presents a short, high level overview of this methodological approach as applied to \gls{WCA}, making very straightforward assumptions about human behavior in \gls{WCA} applications.
\cref{paper:olguinmunoz2019edgedroid} extends the discussion from \cref{paper:olguinmunoz2018demoscaling}, providing more details on the implementation of the necessary measurement framework and the tooling employed.
This work presents the first empirical results obtained with our methodological approach, providing thus an initial validation of its utility for research.
\cref{paper:olguinmunoz2019edgedroid} also discusses in detail the assumptions that were made about human behavior and reactions.
We assume in these works a human which is impervious to poor system performance, and suffers no annoyance, fatigue, frustration, nausea or other shortcomings of real human users.
The result is a model of a user who responds in a precisely reproducible and deterministic manner to the same system stimulus every time.

In \cref{paper:olguinmunoz2022cleave}, we extend our methodology to the study of control systems for \glspl{NCS} on Edge Computing.
These applications are similarly difficult to benchmark as \gls{WCA}, particularly at scale on multi-tenant systems, due to their client-side complexity and extreme sensitivity to latency.
To address these challenges, we introduce a tool, \acs{CLEAVE}, for the emulation and subsequent deployment of these systems on edge computing infrastructure.
This tool allows us both emulate the physical components of a relatively simple control system plant and deploy real algorithms for its control.
The middleware software abstracts away the network from the development of these workloads, enabling quick prototyping and deployment.
We then present use-case scenarios demonstrating the scalability and flexibility of their approach by deploying scenarios with a large number of loops without the need for domain-specific hardware.
This work showcases the applicability of the methodology beyond wearable cognitive assistance and highlights the potential of our approach for other edge computing applications.

\cref{paper:olguinmunoz2022ainur} presents a tangential contribution.
It introduces the software framework used in \cref{paper:olguinmunoz2023realistic,paper:olguinmunoz2022cleave} for the orchestration of the edge computing testbeds on which the developed tools were deployed.
This framework represents an ancillary contribution, crucial for the research presented in these works.
Without it, the experimental approach described in these works would not have been feasible.

Next, in \cref{paper:olguinmunoz2021impact,paper:olguinmunoz2023realistic} we delve deep into the implementation of our methodology for \gls{WCA}, as well as study its implications for optimization of these systems.
Although the above approach to human behavior in \gls{WCA} discussed represents a useful initial approximation, it is nonetheless not a realistic model of it.
In \cref{paper:olguinmunoz2021impact} we therefore take the first major step towards such a model, presenting a deep characterization of human behavior in these applications.
We develop this characterization through a human subject study with a cohort of \num{40} participants who were asked to interact with an instrumented \gls{WCA} application.
System responsiveness is altered in real-time during each execution of the task, and we record participants' reactions by measuring task- and system-related metrics, as well as biometrics from sensors placed on their bodies.
Participants were also asked to fill out two personality indicator questionnaires, allowing us to later correlate individual personality traits and measured reaction to changes in system responsiveness.

\cref{paper:olguinmunoz2023realistic} then concludes this line of work, building upon the insights and data obtained in \cref{paper:olguinmunoz2021impact} to develop the first ever data-driven model of human timings for \gls{WCA}.
The model is validated against previously obtained results, both through simulated, controlled executions and deployments on a real edge computing testbed.
This work also explores potential implications of this model for \gls{WCA} system optimization potential, particularly in the domains of energy consumption and sampling strategies.

\section{Structure of this dissertation}

This dissertation is structured into two parts.
\cref{part:summary} presents a summary of the research, introducing the topic, discussing related work, and highlighting key contributions.
Next, in \cref{part:publications} we present the publications that form the core of this thesis.

\cref{part:summary} is further structured into four main chapters.
\cref{chap:introduction} serves as an introduction to the topic and outlines the key contributions of this work.
It sets the stage for the subsequent chapters by providing a high-level overview of the research problem and its significance.

In \cref{chap:relatedwork}, we present theoretical background and related work.
The chapter provides a comprehensive review of the existing literature, highlighting the current state-of-the-art and identifying any gaps in knowledge.
We discuss context for our research, and establish the foundation for the contributions presented in the dissertation.

\cref{chap:contributions} is the core of \cref{part:summary} and provides a detailed summary of the contributions made in this dissertation.
We discuss methods, tools, and findings that were developed and obtained throughout the course of our research.
We aim to demonstrate the significance of our work and how it contributes to the field.

Finally, \cref{chap:conclusions} concludes the dissertation and outlines avenues for future research.
We summarize the key findings and contributions of the thesis, and briefly discuss the implications of our findings and contributions for the field.
This chapter also identifies open questions and areas for further research, providing opportunities to build on the work presented in this thesis.
