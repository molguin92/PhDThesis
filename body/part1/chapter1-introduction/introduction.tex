The advancement of technology has brought about new computing paradigms and network technologies that have the potential to revolutionize the way we think about and approach computing and communication.
One of the most promising of these is Edge Computing, which seeks to bring computing closer to the end user, reducing latency and increasing reliability.
Another key player in this space is 5G, the latest generation of mobile communication networks that offers higher data rates, lower latency, and improved reliability.
The combination of these two technologies holds great promise for the deployment of latency-sensitive applications, including \glspl{CPS}, \glspl{NCS}, and \gls{WCA} applications, among others.

Despite these advancements, there are still significant challenges in understanding and scaling these systems, particularly in regard to their cyber-physical nature, complexity, and requirements for low latency and high reliability.
The aim of this thesis is to make a contribution to the field by investigating the applications of a methodology for the study of latency-sensitive applications deployed on Edge infrastructure and network technologies such as 5G.
The proposed methodology aims to enhance the accuracy and realism of results related to Edge infrastructure and 5G networks, particularly in regard to network performance.
Our results will contribute to the development of new techniques and approaches for improving the performance and reliability of \gls{MEC}.

Our methodology is based on the emulation of target workloads on actual Edge infrastructure.
We replace the client side of the system with a realistic emulation of the desired behaviors, implemented in software deployed on \gls{COTS} general-purpose computing devices.
In our initial implementation, these correspond to low-cost, easily replaceable, and scalable Raspberry Pi 4 Model B \glspl{SBC}.

Emulating the workload component reduces complexity by moving it into the software domain, allowing for easier scaling through the use of cheap, COTS general-purpose hardware such as \glspl{SBC}.
It also preserves the realism of effects stemming from the hardware and network.
In particular, the methodology allows us to capture effects due to network factors such as contention, congestion control, and medium access, which are often of stochastic or chaotic natures and complex to capture in simulations.

The methodology also provides improved repeatability and replicability.
Repeating a study becomes a matter of re-running the workload on the same testbed, and studies can be replicated simply by obtaining the same or equivalent software workload and deploying it on a comparable testbed.
These are complex tasks to accomplish in real-world approaches, particularly when dealing with humans.

Our methodology provides a comprehensive and realistic assessment of the performance of latency-sensitive applications deployed on Edge infrastructure.
The approach provides valuable insights for researchers, system designers, and application developers, and will contribute to the development of new techniques and approaches for improving the performance and reliability of Edge Computing and 5G systems.
