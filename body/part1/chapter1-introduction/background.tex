\section{Background}

\subsection{Edge computing}
\glsresetall%

Edge computing is a novel distributed computing paradigm, emerging from a need to overcome the drawbacks of offloading computation and data to the cloud.
Cloud computing, the reigning distributed computing model, allows users to access shared pools of resources such as servers, databases, and applications, over the internet~\cite{gai2012towards}.
These pools of resources are managed in a centralized manner by specialized providers, and users and businesses can access them on-demand, without having to invest in and manage infrastructure of their own.
Providers in turn employ economies of scale, providing these services by deploying massive amounts of computing power and storage capacity in specialized locations known as datacenters~\citationeeded.
These hardware resources are then further compartmentalized through the use of virtualization technologies such as \glspl{VM} and containers~\cite{gai2012towards}.

Through this design, cloud computing affords significant advantages to users.
As services are deployed in a centralized manner accessible over the internet, users can interact with their data and applications from anywhere in the world, from any device.
Thanks to economies of scale and virtualization technologies, the cloud is highly scalable;
services can be scaled simply by spawning more \glspl{VM}.
The specialized nature of cloud providers, the scale of modern datacenters, and the use of virtualization also make the cloud highly reliable.
When hardware fails, recovering is simply a matter of migrating the service container or \gls{VM} to an available compute node~\cite{endo2016high}.

However, the cloud is not suitable for everything, and presents important drawbacks and challenges for latency-sensitive and/or bandwidth-intensive applications.
In order to achieve the necessary economies of scale, cloud datacenters are designed to serve users distributed across vast geographical areas.
These installations are thus often located ``far'' from potential users;
for instance, at the time of writing, \gls{AWS} processes traffic from all of Scandinavia and the Baltic countries through a single datacenter in Stockholm~\cite{awsregions}.
This leads to prohibitively high latencies for both highly interactive immersive applications such as mobile \gls{XR} and for \glspl{CPS} and \glspl{NCS}~\cite{tolia2006quantifying,lagar2007interactive,satyanarayanan2009case,varghese2016challenges,shi2016promise}.
The former category requires \emph{motion-to-photon} latencies (i.e.\ time between input capture and feedback) below \SI{60}{\milli\second} for interactions to be perceived as fluid and responsive by the user~\cite{chen2017empirical}; the latter can require sub-\SI{10}{\milli\second} latencies, for instance in the case of vehicular safety systems.
Such latencies are unfeasible to consistently achieve with cloud computing~\cite{dang2021cloudy}.
On the other hand, as smart devices, appliances, and sensors become more and more ubiquitous, the network architectures of modern datacenters face increasing challenges to deal with the massively increasing volume of traffic~\cite{shi2016edge,wang2019towards}.

\begin{figure}
    \centering
    \includegraphics[height=30em]{Figs/edgecomputing}
    \caption{%
        Conceptual design of edge computing.
        Micro-clouds (known as ``cloudlets''~\cite{satyanarayanan2009case}) are placed a at the edge of the network, a few hops away from end users.
        In other words, these cloudlets are located between users and the internet and the cloud.
    }\label{fig:edgecomputing}
\end{figure}

\medskip
Edge computing emerges as a potential answer to these challenges~\cite{satyanarayanan2009case,shi2016promise,shi2016edge,varghese2016challenges,satyanarayanan2017emergence,bittmann2017edge,wang2019towards}.
The foundation for this concept was laid by \citeauthor{satyanarayanan2009case}~\cite{satyanarayanan2009case} in\ \citeyear{satyanarayanan2009case}.
To tackle the prohibitively high latencies of cloud computing, the authors proposed an extension of the existing cloud computing paradigm with compute nodes at the edge of the Internet.
Instead of offloading computation to a datacenter potentially thousands of kilometers away, in edge computing it is offloaded to a micro-datacenter at the \emph{edge} of the network close to the user.
\citeauthor{satyanarayanan2009case} named these compute nodes \emph{cloudlets}, and envisioned them as featuring most of the key characteristics of cloud datacenters, such as multi-tenancy, virtualization of computing resources, virtually unrestricted access to energy, as well as limited scalability, all while being one or two hops away from the user (see \cref{fig:edgecomputing})
This built on previous work on \emph{cyber foraging}~\cite{noble1997agile,flinn1999energy,satyanarayanan2001pervasive}.
Cyber foraging refers to the extension and amplification of the capabilities of mobile and wearable devices by offloading computation and data manipulation to nearby infrastructure (as opposed to distant infrastructure such as the cloud).
This reduces energy consumption and allows for the deployment of otherwise unfeasible workloads on mobile hardware.

The architecture of edge computing offers several advantages, the primary being the significant reduction of latency.
Cloudlets can serve highly latency-sensitive and resource-intensive applications with extremely low latencies while still providing orders of magnitude more computing and energy resources than those present on mobile and wearable devices.
Another advantage is bandwidth reduction.
%By processing and aggregating data closer to where its needed, edge computing can significantly reduce the amount of traffic going to cloud datacenters.
A variant of edge computing called \emph{fog} computing, introduced by \citeauthor{bonomi2012fog}~\cite{bonomi2012fog} in\ \citeyear{bonomi2012fog}, concerns itself specifically with the distribution of computing power between the cloud and the edge for this purpose.
By aggregating, transforming, and filtering data at multiple levels in the network, fog computing aims to reduce the load placed on cloud datacenters by massive \gls{IOT} sensor networks~\cite{yi2015survey}.
Edge computing also presents opportunities for increased data security, privacy, and integrity, by keeping it geographically close to its origin and by allowing for anonymization through denaturing and local aggregation before offloading to cloud services~\cite{satyanarayanan2017emergence}.

Finally, in later years, edge computing has been paired with novel mobile networking standards such as 5G~\cite{hassan2019edge,pham2020survey,wan2020efficient}.
This combination is often referred to as \gls{MEC}, and has the potential to enable unprecedented use-cases requiring stringent latency bounds and high bandwidths.
Two examples of such applications, of relevance for the present work, are \acl{WCA}~\cite{ha2014towards,chen2018application,wang2020scaling,chen2017empirical,chen2018application} and \aclp{NCS}~\cite{sasaki2016vehicle,wang2018bandwidth,wan2020efficient}.
These applications had been hitherto inviable to implement due to the limitations of cloud offloading in mobile and wearable contexts, but have now become a practical reality thanks to \gls{MEC}.
We will discuss these in depth in their respective sections, \cref{background:wca} and \cref{background:ncs}.

\subsubsection{Experimental research in Edge Computing}
\todo[inline]{Should be moved to related work or removed.}

The last few years have seen the emergence of a number of small- to mid-scale platforms and testbeds for edge computing research.
Several of these are additionally driven by research interests in novel mobile networking technologies and \gls{MEC}, as well as in the virtualization of network functions.
Of particular interest to us are the
\begin{inlineenum}
    \item \acs{GENI}
    \item Chameleon
    \item \acs{COSMOS}
    \item \acs{POWDER}
    \item \acs{ARA}
    \item EdgeNet
%    \item Drexel Grid \gls{SDR}
\end{inlineenum} testbeds.

\gls{GENI} is a testbed deployed across more than fifty sites across the \gls{USA}.
Originally designed for ``the development, deployment, and validation of transformative, at-scale concepts in network science, services, and security''~\cite{berman2014geni}, \gls{GENI} today combines local edge compute resources with mobile network last-hop connectivity for \gls{MEC} experimentation~\cite{gosain2017geni}.
The wireless-capable sites incorporate single-hop access to compute, storage, and network resource, making \gls{GENI} a compelling edge computing platform.

The Chameleon testbed is a research platform originally designed for the study and evaluation of cloud computing systems~\cite{keahey2020lessons}, but which has recently been expanded to target edge computing research through the \acs{CHI}@Edge project.
Chameleon is a collaborative project operated by a consortium of research institutions and industry partners, and is designed to provide a flexible and reconfigurable infrastructure for researchers to experiment with and test various cloud computing technologies, algorithms, and configurations.
The testbed provides a highly reconfigurable array of hardware and software resources, including servers, storage systems, and networking equipment, able to simulate various cloud and edge computing scenarios.

\gls{COSMOS} is a testbed deployed in New York City (State of New York, \gls{USA}), containing over \num{200} rooftop, intermediate, and mobile nodes.
The testbeds main \emph{raison d'etre} is \gls{MEC} research, with a strong emphasis on wireless networks.
The system includes \glspl{SDR}, \si{\milli\meter}-wave equipment, optical fibers, cloud integration, and edge compute for core network functionality and application data processing~\cite{yu2019cosmos,raychaudhuri2020challenge}.

\gls{POWDER} is described by the authors of~\cite{breen2020powder} as a ``city-scale, remotely accessible, end-to-end software defined platform supporting a broad range of wireless and mobile related research''.
It features roughly \num{15} fixed programmable radio nodes, based on off-the-shelves \glspl{SDR}, distributed across a \SI{15}{\kilo\meter\squared} area in Salt Lake City (State of Utah, \gls{USA}).
These radio nodes are combined with edge-based compute nodes as well as cloud resources for full \gls{MEC} experimentation.

The \gls{ARA} platform~\cite{zhang2022ara} is a highly flexible testbed based on the \gls{CHI} software suite originally developed for the aforementioned Chameleon testbed~\cite{keahey2020lessons}.
\gls{ARA} is developed by researchers at Iowa State University in Ames, Iowa (\gls{USA}), and spans a rural area with a diameter of over \SI{60}{\kilo\meter}~\cite{zhang2022ara}.
Its core goal is the study and deployment of advanced edge computing and wireless platforms and technologies in real-world agricultural and rural settings, and includes a broad range of wireless technologies deployed through both \glspl{SDR} and programmable \gls{COTS} radios, as well as automated ground vehicles, cameras and sensors.

EdgeNet deviates from the approaches described above and represents a fully software-based, massively distributed public platform for edge computing research~\cite{cappos2018edgenet,senel2021edgenet1,senel2021edgenet2}.
It consists of a public, modified and customized Kubernetes~\cite{kubernetes} cluster.
The EdgeNet software is available for anyone to download;
individuals and organizations can quickly deploy self-hosted edge nodes which are then made available to anyone using the platform.
The end result is a flexible and scalable global edge cloud simplifying research and prototype development in edge and cloud computing.
A similar, closely-related project is KubeEdge~\cite{xiong2018extend}, a software framework for extending Kubernetes clusters from the cloud to the edge.

Finally, a number of other, small-scale approaches also exist.
In~\cite{gedawy2016cumulus}, the authors propose Cumulus, a prototype testbed for edge compute offloading in \gls{IOT} with the explicit goal of providing a generic and heterogeneous platform for edge computing research.
The authors of~\cite{rimal2018experimental} propose an experimental testbed for two-tiered edge architectures in the context of \gls{FIWI} edge computing.
This testbed is then employed to validate wireless access and compute resource scheduling strategies in these systems.
In~\cite{yamanaka2021design}, the authors present an experimental edge computing testbed capable of determining the offload location of a workload based on desired latency parameters.
\cite{diao2019scalable} propose yet another Kubernetes-based testbed based for edge task offloading based on heterogeneous hardware including low-power single-board computers such as Raspberry Pis and Jetson Nanos.
Finally,\ \cite{moorthy2022cloudraft} proposes \gls{CLOUDRAFT}, a cloud-based framework for mobile network experimentation, with a focus on simplifying the management of testbed resources.
The goal of this project is to integrate, coordinate, share, and improve upon existing testbeds through a common interface.


\subsection{\glsfmtlong{XR}}\label{background:xr}
\todo[inline]{Rework into background XR}

\glsreset{WCA}
\gls{WCA} refers to a novel category of wearable and mobile immersive applications which aim to amplify human cognition in both day-to-day tasks and professional settings.
\glspl{WCA} work in a mode analogous to how \gls{GPS} navigation systems guide drivers, by seamlessly providing relevant instructions and feedback relating to the current task at hand and staying out of the way of the user when guidance is not required.
These applications operate much like a human assistant would, by observing the performance of the user and offering guidance proactively.
Through the use of sensors (most commonly video inputs), the application follows the progress of the task in realtime by continuously sampling the state of the physical system.
The input is parsed to an internal symbolic representation of the state of the task;
whenever a change of state is detected the application provides appropriate feedback or instructions to the user.
This feedback can be in the form of text, images, video, and/or audio, and guides the user towards the next desired system state.
If no change in state is detected, the application remains silent and out-of-the-way.
In other words, the application silently discards samples which capture an intermediate or unfinished state, or simply are too noisy to be processed.
\Cref{fig:wca} illustrates this mode of operation in a \gls{WCA} guiding a user to manipulate a structure composed of LEGO bricks.

\begin{figure}
    \centering
    \includegraphics[width=.9\textwidth]{Figs/wca_state}
    \caption{%
        Mode of operation of \acs{WCA} applications.
        The assistant continuously samples the state of the monitored task, and provides appropriate feedback once a state change has been detected.
        Samples that do not result in feedback are silently discarded.
    }\label{fig:wca}
\end{figure}

As the name implies, \glspl{WCA} are available whenever the user requires them, without being tethered to a particular physical location;
they are pervasive and mobile~\cite{ha2014towards}.
This leads to interesting consequences for \gls{WCA} system design when combined with their requirement of seamless integration with the context of the user.
The wearable nature of \gls{WCA} directly implies use of lightweight and battery-powered devices, greatly constrained in terms of energy and computing resources.
The requirements of immersive and seamless interaction, on the other hand, suggest a level of context sensitivity and proactivity that can only be provided by real-time analysis of sensor inputs such as video and audio feeds.
This kind of compute- and energy-intensive processing, often involving \gls{ML} algorithms and \glspl{DNN}, is unfeasible to be performed on wearable devices, and thus these applications \emph{must} necessarily make use of compute offloading~\cite{ha2014towards,wang2020scaling}.

However, it is understood today the cloud represents an unsuitable candidate for offloading computation in these applications, given the high latencies involved.
Given their immersive nature, feedback in \gls{WCA} systems should be provided ``quickly'' (relative to the task at hand), as users will have expectations of constant and immediate feedback as they interact with the system.
As is the case for most \gls{XR} applications, delayed feedback in \gls{WCA} can have severe negative consequences for the quality of the user experience.
Depending on the task, these consequences can range from simply distracting or annoying the user (in the case of less interactive tasks such as step-by-step assembly), to actively handicapping user performance (in the case of highly interactive tasks).
There is therefore a broad understanding today that edge computing is a key enabling technology for \gls{WCA} applications~\cite{ha2014towards,wang2020scaling,chen2018application,olguinmunoz2021impact}.

\medskip
In literature, research on \gls{WCA} has focused mostly on the implementation of prototype applications and platforms.
\citeauthor{ha2014towards}~\cite{ha2014towards} coined the term \acl{WCA} in\ \citeyear{ha2014towards}.
Their work described a first prototype system employing a Google Glass wearable device, and identified the challenges of offloading the computation for such an application to the cloud.
The authors were originally inspired by assistive use cases for people suffering from some form of cognitive decline due to aging, illness, or traumatic brain injury~\cite{ha2014towards,satyanarayanan2019augmenting}.
More recently, the scope of \glspl{WCA} has been expanded to include a broader range of use cases, including complex assembly tasks~\cite{chen2017empirical,chen2018application,wang2020scaling,wang2019towards}.
\cite{chen2015early} and~\cite{chen2018application} further developed the platform described in~\cite{ha2014towards}, and produced a number of different assistive applications such as the aforementioned LEGO assistant, and others such as a Ping-Pong assistant capable of providing real-time hints to players of table tennis, and more advanced assembly assistant for IKEA furniture.
Yet another platform for \gls{WCA} was introduced in~\cite{chatzopoulos2017hyperion}, with the goal of providing real-time contextual information in text-form to users.
Furthermore, important work has been done in the implementation of such systems for industrial and factory applications~\cite{belletier2021wearable}.
Non-wearable cognitive assistance systems for assembly tasks have already been proven to be valuable tools in the industrial workplace~\cite{funk2015cognitive,gorecky2011cognito}.
It is understood that the detethering of these system for their use in wearable scenarios opens up a multitude of possibilities.

In later years, a body of work studying the optimization of these applications has emerged.
A seminal work in this field is~\cite{chen2017empirical}, in which the authors establish hard and soft latency bounds for acceptable performance in \gls{WCA}.
\citeauthor{wang2019towards}~\cite{wang2019towards} explore strategies for scalable \gls{WCA} on the edge through different adaptive sampling and resource allocation schemes;
this is further developed in the authors' PhD thesis later on~\cite{wang2020scaling}.
\cite{moothedath2021energy},\ \cite{moothedath2022energy1}, and~\cite{moothedath2022energy2} studied the potential for energy optimization in \gls{WCA} through convex optimization of the sampling strategies employed.
In~\cite{george2020openrtist}, the authors introduce \emph{OpenRTiST}, a benchmarking tool for immersive, latency-sensitive, and bandwidth-hungry applications on edge computing leveraging \glspl{DNN} to perform real-time style transfer on a live video feed.

The present thesis aims to contribute to this body of work by presenting and discussing a viable methodology for the design and evaluation of these systems.
There's more to end-to-end latency, \gls{WCA} performance, and scalability than merely the question of where the compute backend is placed, however.
The design of any complex application such as \gls{WCA} involves a multitude of decisions with the potential to influence the system responsiveness as experienced by the user.
These decisions include those on the implementation side (compression standards, algorithms, protocols, etc.), as well as on the infrastructure side (physical network layer, traffic prioritization, etc.).
As expressed above, existing studies of this class of applications have only recently started to delve more deeply upon these issues.
On the other hand, recently published models for end-to-end latency of edge computing architectures, are quite complex, while not accounting for the specifics of \gls{WCA} and in many cases limited to very constrained scenarios and analyses~\cite{al_zubaidy2015performance,schiessl2017finite}.
Our work in~\cite{olguinmunoz2018demoscaling,olguinmunoz2019edgedroid} introduces the first ever model for realistic emulation of these applications on real hardware.
Our characterization of human timings in these systems in~\cite{olguinmunoz2021impact}, and our subsequent design of a realistic model for human behavior in \gls{WCA} constitute novel insights and tools for the optimization of these systems.

\subsection{\glsfmtlongpl{NCS}}\label{background:ncs}
\glsreset{NCS}
\todo[inline]{Needs citations. Remove?}

\glspl{NCS} are a type of control system that involves the integration of communication networks into traditional control systems.
A control system is a collection of devices or subsystems that work together to achieve a desired behavior or outcome of a physical system.
Control systems are composed of
\begin{inlineenum}
    \item a \emph{plant}
    \item \emph{sensors}
    \item \emph{actuators}
    \item a \emph{controller}
\end{inlineenum}.
Plant refers to the physical system being controlled.
Sensors sample and encode the state of the plant and transmit it to the controller.
The controller processes these inputs, and generates control signals to be sent to the actuators, which modify the plant's behavior.
The overall goal of a control system is to regulate a process or a physical system to meet desired specifications or objectives, such as maintaining a constant temperature, tracking a set trajectory, or regulating the speed of a machine.

\glspl{NCS}~\cite{gupta2010networked} aim to enhance and extend the capabilities of traditional control systems by decoupling the controller from the plant, sensors, and actuators and interconnecting them with general-purpose \acs{TCP}/\acs{IP} networks;
see \cref{fig:csvsncs}.
This allows for remote monitoring and control, which enhances the reliability and safety of the control system, which is particularly useful in applications where real-time monitoring and control are critical, such as in industrial automation and transportation systems.
\glspl{NCS} also enable the integration of multiple control systems, as well as centralized control of multiple plants by a single controller.
This leads to enhanced scalability and flexibility, and allows improved coordination and collaboration among systems.
Furthermore, \glspl{NCS} provide greater access to information, as they enable the exchange of data and information between control components and systems, leading to more informed decision-making and improved performance.

\begin{figure}
    \centering
    \begin{subfigure}[t]{0.45\textwidth}
        \centering
        \includegraphics[width=\textwidth]{Figs/control_system}
        \caption{%
            Traditional control system.
        }
    \end{subfigure}%
    \hfill%
    \begin{subfigure}[t]{0.50\textwidth}
        \centering
        \includegraphics[width=\textwidth]{Figs/networked_control_system}
        \caption{%
            \acl{NCS}.
        }
    \end{subfigure}
    \caption{%
        Comparison between traditional control systems and \aclp{NCS}.
        In an \aclp{NCS}, controller and plant are physically separated and both actuation commands and sensor inputs are shared through a general purpose \acs{TCP}/\acs{IP} network.
    }\label{fig:csvsncs}
\end{figure}

\glspl{NCS} have become increasingly popular in recent years due to advances in communication and control technologies, as well as the increasing demand for real-time, remote, and distributed control.
They have found use in varied applications, such as industrial automation, transportation systems, and smart buildings, and some have attempted to leverage the cloud for centralized, distributed control in what has been called \emph{cloud control systems}~\cite{xia2015cloud}
However, depending on the physical system being controlled, \glspl{NCS} can have stringent timing and reliability requirements for communication that conventional cloud paradigms simply cannot meet~\cite{wan2020efficient}.
This has led their adoption to be limited mostly to industrial environments.
This is about to change, however, as with the advent of edge computing, paired with novel wireless communication technologies such as cellular 5G, consumer-grade \glspl{NCS} will be made possible.
These technologies will likely become the backbone of consumer-grade, widely-deployed \glspl{NCS}, enabling real-time capabilities through extremely low end-to-end latencies together with context- and locality-awareness.

\medskip
Research in \glspl{NCS} has mostly focused on addressing the challenges imposed by the introduction of best-effort communication networks into traditional control systems.
Network delays (latency), losses, and \emph{jitter} --- referring to the variation in random network delays --- are at the forefront of these challenges.
They have partially been addressed through the clever placement of the controller (e.g.\ by employing edge computing)~\cite{sasaki2017layered,sasaki2016vehicle}, as well as through the development of control algorithms capable of accounting for the effects of the network~\cite{zhang2013network}.

In terms of research relating to the benchmarking and optimization of these systems, works in literature have for the most part leveraged either theoretical models~\cite{zhang2019networked} or simulations (for example ~\cite{ma2019optimal}).
However, these approaches can only capture network behaviors at a coarse level, and thus a body of works has emerge in later years involving experimental approaches to \gls{NCS} benchmarking.
\cite{baumann2018evaluating} and~\cite{cuenca2019periodic} employ purpose-built physical testbeds for the study of these systems.
\cite{zoppi2020ncsbench} introduces \emph{NCSBench}, a platform for reproducible benchmarking of \gls{NCS} using a LEGO Mindstorms platform.

Finally, hybrid approaches, mixing testbed or prototype implementations and simulations or emulations of \gls{NCS} components also exist in literature.
These approaches tend to focus on a particular element of the system which is implemented using a prototype or testbed, and virtualize the rest of the system.
For instance, in~\cite{wang2020inverter} the authors are concerned with realistic effects from the network, and thus they have an emulated \gls{NCS} interact with a real network.
On the other hand, works such as~\cite{natale2004inverted} focus on the system being controlled and opt to instead emulate the network.

%\glspl{NCS}~\cite{Gupta2010NCSOverview}, a type of \gls{CPS} wherein multiple networked actuators and sensors form a part of the same automatic control system will benefit from the adoption of these technologies.
%Depending on the physical system being controlled, \glspl{NCS} can have stringent timing and reliability requirements for communication that conventional cloud paradigms and cellular networks cannot meet~\cite{Wan2020Efficient}.
%This necessitates sophisticated tools for the performance evaluation of future system architectures, as well as novel NCS design paradigm.

%Due to their potential advantages for industrial and commercial settings, there exist works~\cite{Zhang2016Survey} dedicated to the modelling and performance characterization of \glspl{NCS}, improving NCSs by distributing control functions across networks, facilitating centralized coordination, control, and monitoring.

%One the one hand, related literature in \glspl{NCS} leverages to a large extent theoretical models, at the price of being able to capture networked systems effects only on a coarse level.
%%follows a theoretical approach, and only a small fraction of it deals with experimental studies.
%On the other, there exist several approaches when considering experimental methodologies.
%%A number of works concerning \glspl{NCS} deal with experimental studies.
%%\glspl{NCS} have an inherently inter-domain nature intertwining knowledge from the fields of communications, computing, and control theory in ways that cannot be studied in isolation, leading to various different approaches to such studies.
%One such approach uses setups in which the complete system is built on top of real hardware.
%This approach is employed in the works of Baumann \emph{et al.}~\cite{Baumann2018LowPower} and Cuenca \emph{et al.}~\cite{Cuenca2019UAV}; in both of these, the authors implement their approach on physical testbeds.
%Conversely, other studies choose to instead use completely \emph{simulated} \gls{NCS} setups.
%The authors in\ \cite{Ma2019DynamicSched} have opted for such an approach.
%These studies often employ combinations of physical and network simulation tools trying to capture the complex dynamics of \glspl{NCS}.
%Finally, some experimental studies instead employ \emph{virtualized} approaches, in which either
%\begin{inlineenum}[itemjoin={{; }}, itemjoin*={{; or }}]
%    \item a real network interacts with a simulated or emulated control system~\cite{Wang2020VoltageControl}
%    \item a simulated network interacts with a real control system~\cite{Natale2004InvPendEthernet}.
%\end{inlineenum}
%
%As evidenced above, experimental research in \glspl{NCS} includes varied heterogeneous hardware and software platforms, methodologies and key performance indicators.
%This, in turn, leads to hardware, software, and methodology fragmentation, as different studies tend to prefer approaches more favored in their respective communities.
%Furthermore, existing studies tend to focus on individual aspects and components of a system, thus producing results which do not provide a complete image of the \gls{NCS}.
%This has caused a gap in knowledge pertaining to the reproducibility and comparison of experimental studies on these systems.
%
%Zoppi \emph{et al.}~\cite{Zoppi2020NCSBench} made the first (and to the best of our knowledge, the only) attempt at tackling this challenge in their work.
%In their work, they proposed a platform called NCSBench, to be used for reproducible benchmarking in NCS.\
%Their methodology utilizes joint knowledge of control, computation, and communication.
%In their work various architectural elements and the corresponding delays associated with the NCS are modelled.
%Multiple experimental parameters and certain observable key performance indicators are defined and utilized in the implementation.
%This work however utilizes a physical LEGO\textregistered{}\ Mindstorms EV3 Core Set\texttrademark{}\  based plant for the implementation, preventing instantaneous changes in plant characteristics and component parametrizations.
%Furthermore, relying on physical objects like an inverted pendulum limits scalability of the experimentation in practice.

\subsection{Emulation as a research methodology}
\todo[inline]{Rename?}
