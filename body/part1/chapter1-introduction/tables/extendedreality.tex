% Please add the following required packages to your document preamble:
% \usepackage{booktabs}
\begin{table}[]
    \centering
    \caption{Comparison between different \acs{XR} technologies.}
    \label{tab:xrcomparison}
    \tiny
    \renewcommand{\arraystretch}{1.5}
    \begin{tabularx}{\textwidth}{@{}lYYY@{}}
        \toprule
        &
        \textbf{\acl{VR}} &
        \textbf{\acl{AR}} &
        \textbf{\acl{MR}} \\ \midrule
        % \textbf{Definition} &
        % A technology that creates a completely digital environment that simulates a real-world experience &
        % A technology that superimposes digital objects on the real world &
        % A technology that combines digital objects with the real world in a seamless way \\
        \textbf{Immersiveness} &
        High &
        Medium &
        Medium to high \\
        \textbf{User Interaction} &
        Only with the digital world &
        Both with the real and digital worlds &
        Both with the real and digital worlds in a seamless way \\
        \textbf{Equipment Required} &
        Headset and controllers &
        Smartphones, tablets, or \acs{AR} glasses &
        Headset and controllers \\
        % \textbf{Use Cases} &
        % Gaming, education, and training &
        % Retail, entertainment, education, and gaming &
        % Industrial design, engineering, and military training \\
        \textbf{Advantages} &
        Provides realistic immersive experience in a completely digital environment that can be easily adapted and changed. &
        Can be implemented cheaply on massively-available technology such as smartphones. &
        Provides a combination of AR and VR benefits, allowing seamless interaction between real and virtual worlds. \\
        \textbf{Drawbacks} &
        Isolates users from the real world, may cause motion sickness, and requires expensive equipment. &
        Limited field of view, reliance on external environment, and not fully immersive. &
        Expensive, complex, and may cause visual discomfort \\
        \textbf{\acs{QoS} Requirements} &
        High bandwidth, low latencies.
        Delay and jitter can lead to motion sickness. &
        Low latencies to ensure accurate, real-time overlay of digital assets. &
        High bandwidth, low latencies.
        Delay and jitter can lead to motion sickness. \\
        \bottomrule
    \end{tabularx}
\end{table}