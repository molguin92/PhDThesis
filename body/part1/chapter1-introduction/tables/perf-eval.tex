\begin{table}[]
    \centering
    \caption{Comparison between the three main approaches to performance evaluation of networked systems.}\label{tab:performance-evaluation}
    \tiny
    \renewcommand{\arraystretch}{1.5}
    \begin{tabularx}{\textwidth}{@{}lXXX@{}}
        \toprule
        & \textbf{Analytical modeling} & \textbf{Simulations} & \textbf{Experimental research} \\
        \midrule

        \textbf{Characteristics} &
        Based on mathematical models. &
        Based on mathematical models implemented as software to simulate the behavior of the system over time. &
        Involves direct measurement of system performance under real-world conditions. \\

        \textbf{Advantages} &
        Efficient. Provides theoretical insights and understanding of the system. &
        More realistic representation of the system's behavior and performance.
        Easy to extend and combine with other models for additional realism.
        Allows for controlled and repeatable experimentation. &
        Provides the most accurate assessment of system performance.
        Able to capture factors that cannot be modeled or simulated, such as non-linearities and stochastic or chaotic factors. \\
            
        \textbf{Disadvantages} &
        Limited in complexity, accuracy, and realism.
        May be unable to account for all relevant real-world factors. &
        Limited by the accuracy of the models used and the assumptions made.
        May not capture all relevant factors.
        Often computationally intensive and time-consuming. &
        Can be expensive, time-consuming, and difficult to replicate.
        When involving humans, ethical concerns must be addressed.
        May not be feasible for all research questions. \\
        
        \bottomrule
    \end{tabularx}
\end{table}