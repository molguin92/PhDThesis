\section{Structure of this dissertation}

This dissertation is structured into two parts.
\cref{part:summary} presents a summary, introducing the topic, discussing related work, and highlighting key contributions.
Next, in \cref{part:publications} I present the publications that form the core of this thesis.

\cref{part:summary} is itself structured into four main chapters.
\cref{chap:introduction} serves as an introduction to the topic and outlines the key contributions of this work.
It sets the stage for the subsequent chapters by providing a high-level overview of the research problem and its significance, as well as provides background information on the core topics of the thesis.
In \cref{chap:relwork}, I present the scope of the thesis, as well as relevant related work.
The chapter provides a comprehensive review of the existing literature, highlighting the current state-of-the-art and identifying any gaps in knowledge.
I discuss context for my research, and establish the foundation for the contributions presented in the dissertation.
\cref{chap:contributions} is the core of \cref{part:summary} and provides a detailed summary of the contributions made in this dissertation.
I discuss methods, tools, and findings that were developed and obtained throughout the course of my research.
I aim to demonstrate the significance of my work and how it contributes to the field.
Finally, \cref{chap:conclusions} concludes the dissertation and outlines avenues for future research.
I summarize the key findings and contributions of the thesis, and briefly discuss the implications of my findings and contributions for the field.
This chapter also identifies open questions and areas for further research, providing opportunities to build on the work presented in this thesis.