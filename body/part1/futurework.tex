\section{Conclusions}

\section{Open questions \& future work}

The present thesis presents a comprehensive exploration into the application of our methodology for \glspl{NCS} and specially for \gls{WCA}.
Nonetheless, a number of key questions remain open for future research in this field.

The first of these relates to the applicability of the methodology to other systems and applications in an Edge Computing context.
In this thesis, we have delved deep into its applicability for step-based \gls{WCA} and performed a shallower case study into its advantages for \gls{NCS} research.
The question remains whether our methodology is suitable for
\begin{inlineenum}
%    \item other categories of \gls{WCA} which do not follow a pre-defined, linear task model (e.g.\ \citeauthor{chen2018application}s Ping-Pong assistant application)
%    \item more complex \glspl{NCS}, for instance involving more complex controllers and system architectures
    \item other, non-\gls{WCA} \gls{XR} applications, such as more general mobile \gls{AR} and \gls{VR}
    \item other more general categories of \glspl{CPS}
\end{inlineenum}.
The results we have so far obtained would indicate that our methodology is in fact applicable to a broad spectrum of interactive and \gls{CPS} applications on Edge Computing, but these results should be validated by domain-specific studies.

In line with the more general questions posed above, we identify a number of more specific questions related to the individual contributions to literature made by this thesis.
Beginning with the tool, EdgeDroid, introduced in \cref{paper:olguinmunoz2018demoscaling,paper:olguinmunoz2019edgedroid,paper:olguinmunoz2023realistic}, we believe there are opportunities for its extension toward other classes of \gls{WCA} applications.
Our current model only targets a particular class of step-based \gls{WCA} with a linear task model, but our tool --- and by extension, our methodology --- could be applied to similar applications without much difficulty.

In that same vein, our data~\cite{olguinmunoz:impact2021} only considers young undergraduate students at a highly competitive university in the United States.
An extension of this dataset and the model of human behavior developed in \cref{paper:olguinmunoz2023realistic} towards a more representative sample of the general population would surely be a valuable endeavor.
Such research could also determine new individual difference factors that would substantially impact response to \gls{WCA} in novel populations, something which would represent a valuable contribution to literature in \gls{HCI} in general.

The optimization framework employed in \cref{paper:olguinmunoz2023realistic} to study the implications of our methodology for the optimization of \gls{WCA} could be improved by expanding it towards distributions which better describe the behavior of human execution times.
Currently, the framework assumes executions times are Rayleigh-distributed, however in literature these values are generally understood to follow an \gls{exGaussian} distribution.
Moreover, the utility of the framework could be significantly increased by considering extensions towards scenarios with multiple clients and a distributed solution.

\medskip
Opportunities for future research exist also in regard to our work on applying the methodology to \gls{NCS}.
Currently, \gls{CLEAVE} includes a small repertoire of emulations and controllers.
We are planning to extend this with the goal of creating an open library of \glspl{NCS} to share with the community.


\todo[inline]{Finish talking about CLEAVE.}

\todo[inline]{Then talk about tremendous opportunities for EXPECA and Ainur. Talk about future of EXPECA.}

At the moment, the interactions of \gls{CLEAVE} and tools such as Docker are still quite superficial.
Our goal is to achieve a much tighter integration, e.g.\ by providing the toolkit as pre-packaged container images.
Finally, the validity of the results obtained by the framework will have to be verified through more thorough, realistic scenarios than what we have been able to show in this work.
In particular, we intend to perform large-scale experimentation targetting 5G cellular deployments, as this technology is set to become the backbone of edge networks in the near future.







Extending methodology to other classes of \gls{WCA}.
Ping-pong is the typical example --- look for other classes too.

Extending methodology to other systems.
\gls{AR}, \gls{XR} in general?
Closed-loop control? (Although we already started with CLEAVE).

Further optimization of \gls{WCA} systems.
In this thesis we touch upon sampling, but other dimensions can also be optimized.
Resource allocation, scheduling, user identification and classification, QoS/QoE guarantees?
