\section{Related work}

\subsection{Benchmarking and emulation of \glsfmtshort{XR} systems}

\textcite{munro2016aaremu}: Software framework for the emulation of AR on Android devices.
A trace of pre-recorded inputs (video, gyro, GPS) are replayed to the real processing code by mocking Android SDK APIS.
Recordings can be ``interactive'', in which case the trace corresponds to images indexed by directional bearings, i.e.\ a 3D sphere), or non-interactive, in which case the trace corresponds to a video replayed at a customizable framerate.

\textcite{williams2013transform}: Toolkit for visualization of AR and evaluation of tracking algorithms.
Recorded trace is replayed in a virtualized environment.


\textcite{berning2013parnorama}: Inverted emulation.
AR application is emulated, user is real.
AR application emulation corresponds to 360 degree FOV video.
Used for prototyping user interactions in AR.

\textcite{choi2022emulating}: XREmul, input emulation tool for XR.
Similar to \cite{munro2016aaremu}, traces of inputs are pre-recorded and then replayed.
Allows for abstraction of inputs in general classes for device-agnostic inputs.
Paper highlights the challenges of deterministic trace (leaves solution to future work).

\textcite{chetoui2022arbench}: ARBench, a benchmarking suite for AR.
Corresponds to a software package including a multitude of AR workloads, emulating the load placed on the the hardware.
Only considers locally-running AR, not network.
Does not emulate human behavior, uses traces of AR applications.

\textcite{george2020openrtist}: OpenRTiST, benchmarking workload for Edge Computing, AR
Based on stable diffusion. 


\subsection{Experimental research in \glsfmtshortpl{NCS}}

A significant amount of work has been dedicated to the modeling and performance characterization of \glspl{NCS} due to their potential benefits for industrial and commercial settings~\cite{lu2016real,hespanha2007survey,zhang2013network,zhang2016survey}.
Research in \glspl{NCS} has mostly focused on addressing the challenges imposed by the introduction of best-effort communication networks into traditional control systems.
Network delays (latency), losses, and \emph{jitter} --- referring to the variation in random network delays --- are at the forefront of these challenges.
They have partially been addressed through the clever placement of the controller (e.g.\ by employing edge computing)~\cite{sasaki2017layered,sasaki2016vehicle}, as well as through the development of control algorithms capable of accounting for the effects of the network~\cite{zhang2013network}.

Most of the large literature concerning \glspl{NCS}, however, follows a theoretical approach, with only small fraction of it dealing with experimental studies~\cite{zhang2019networked}.
\gls{NCS} studies are complex due to the inter-domain nature of these systems, requiring expertise in the fields of communication networks, computing, and control theory.
However, theoretical approaches can only capture network behaviors at a coarse level, and thus a body of works has emerge in later years involving experimental approaches to \gls{NCS} benchmarking.
Within this body of work, we identify three distinct approaches: fully practical, fully simulated, and emulated approaches that combine components of the first two.

In a fully practical approach, the experimental setup uses real systems and hardware to test and validate the performance of the \gls{NCS}.
This approach is straightforward to implement, and can be very useful for evaluating the real-world applicability of the system, but can be costly and time-consuming.
Examples of the application of this approach can be found in studies such as~\cite{drew2005networked,baumann2018evaluating,li2014wireless,cuenca2019periodic}.
These studies implemented wireless and microcontroller-based systems for control and validated them on a physical prototype.

In~\cite{drew2005networked}, the authors present a \gls{NCS} model which considers both random packet delay and loss on both the sensor and actuator sides of the feedback loop.
They validate their design on a physical testbed consisting of inverted pendula controlled from a central computation point over Wi-Fi.
A similar testbed is employed for validation in~\cite{baumann2018evaluating}, in which the authors develop an evaluation methodology for wireless \gls{NCS}.\@
\cite{li2014wireless} implements a wireless microcontroller-based system for vibration control and validates it on a physical prototype comprising a cantilever beam controlled using an \acs{IEEE} 802.15.4 \gls{WPAN} system.\@ \cite{cuenca2019periodic} discusses the design and implementation of a periodic event-triggered sampling and dual-rate control techniques for wireless control, and validate their contributions on a four-rotor autonomous drone platform controlled over WiFi.
\emph{NCSbench}~\cite{zoppi2020ncsbench}, an open-source \gls{NCS} benchmarking platform designed with reproducibility in mind built using the \citetitle{LEGOMindstormsEV3}~\cite{LEGOMindstormsEV3} platform, is another such example.

Completely simulated setups, on the other hand, rely on computer simulations to evaluate the \gls{NCS}'s performance.
This approach is less expensive and faster than the practical approach but may not fully capture the real-world complexities of the system.
Examples of tools and frameworks for the implementation of this approach can be found in works such as~\cite{andersson2005simulation,eyisi2012ncswt}.

A simulated approach is employed in~\cite{du2009novel}, in which the authors develop a novel \emph{Smith} predictor to compensate for varying latencies in wireless \glspl{NCS} and validate it using the \emph{TrueTime}~\cite{henriksson2002truetime} simulator.
In~\cite{chen2015synchronous} the authors validate and evaluate a number of synchronous control strategies for three-motor setups using completely simulated \glspl{NCS} environments.\@
\cite{wu2012application} introduces the concept of \emph{network predictive control} in and uses it to balance a simulated inverted pendulum over a simulated wireless network.\@
\cite{ma2019optimal} proposes an optimal dynamic scheduling strategy that optimizes performance of multi-loop control systems and validate it by simulating a four-loop control system over an \acs{IEEE} 802.15.4 \gls{WPAN}.

Finally, network and hardware-in-the-loop approaches, which aim to strike a balance between realism and efficiency, have also been employed in literature.
Network-in-the-loop refers to experimental setups in which the \emph{network} is instead simulated or emulated.
Such approaches have, for instance, been used to study the effects of the \gls{CSMA/CD} medium access control mechanism on plant stability, as in~\cite{natale2004inverted}.
Hardware-in-the-loop approaches, referring to \gls{NCS} setups in which a real network interacts with a simulated or emulated control system, are particularly prevalent in the field of \emph{smart grid} control.
An example of the application of such an approach can be found in~\cite{wang2020inverter}, in which the authors introduce and validate a novel three-level coordinated control method for photovoltaic inverters.

\subsection{Testbed research in Edge Computing}

The last few years have seen the emergence of a number of small- to mid-scale platforms and testbeds for edge computing research.
Several of these are additionally driven by research interests in novel mobile networking technologies and \gls{MEC}, as well as in the virtualization of network functions.
Of particular interest to us are the
\begin{inlineenum}
    \item \acs{GENI}
    \item Chameleon
    \item \acs{COSMOS}
    \item \acs{POWDER}
    \item \acs{ARA}
    \item EdgeNet
%    \item Drexel Grid \gls{SDR}
\end{inlineenum} testbeds.

\gls{GENI} is a testbed deployed across more than fifty sites across the \gls{USA}.
Originally designed for ``the development, deployment, and validation of transformative, at-scale concepts in network science, services, and security''~\cite{berman2014geni}, \gls{GENI} today combines local edge compute resources with mobile network last-hop connectivity for \gls{MEC} experimentation~\cite{gosain2017geni}.
The wireless-capable sites incorporate single-hop access to compute, storage, and network resource, making \gls{GENI} a compelling edge computing platform.

The Chameleon testbed is a research platform originally designed for the study and evaluation of cloud computing systems~\cite{keahey2020lessons}, but which has recently been expanded to target edge computing research through the \acs{CHI}@Edge project.
Chameleon is a collaborative project operated by a consortium of research institutions and industry partners, and is designed to provide a flexible and reconfigurable infrastructure for researchers to experiment with and test various cloud computing technologies, algorithms, and configurations.
The testbed provides a highly reconfigurable array of hardware and software resources, including servers, storage systems, and networking equipment, able to simulate various cloud and edge computing scenarios.

\gls{COSMOS} is a testbed deployed in New York City (State of New York, \gls{USA}), containing over \num{200} rooftop, intermediate, and mobile nodes.
The testbeds main \emph{raison d'être} is \gls{MEC} research, with a strong emphasis on wireless networks.
The system includes \glspl{SDR}, \si{\milli\meter}-wave equipment, optical fibers, cloud integration, and edge compute for core network functionality and application data processing~\cite{yu2019cosmos,raychaudhuri2020challenge}.

\gls{POWDER} is described by the authors of~\cite{breen2020powder} as a ``city-scale, remotely accessible, end-to-end software defined platform supporting a broad range of wireless and mobile related research''.
It features roughly \num{15} fixed programmable radio nodes, based on off-the-shelves \glspl{SDR}, distributed across a \SI{15}{\kilo\meter\squared} area in Salt Lake City (State of Utah, \gls{USA}).
These radio nodes are combined with edge-based compute nodes as well as cloud resources for full \gls{MEC} experimentation.

The \gls{ARA} platform~\cite{zhang2022ara} is a highly flexible testbed based on the \gls{CHI} software suite originally developed for the aforementioned Chameleon testbed~\cite{keahey2020lessons}.
\gls{ARA} is developed by researchers at Iowa State University in Ames, Iowa (\gls{USA}), and spans a rural area with a diameter of over \SI{60}{\kilo\meter}~\cite{zhang2022ara}.
Its core goal is the study and deployment of advanced edge computing and wireless platforms and technologies in real-world agricultural and rural settings, and includes a broad range of wireless technologies deployed through both \glspl{SDR} and programmable \gls{COTS} radios, as well as automated ground vehicles, cameras and sensors.

EdgeNet deviates from the approaches described above and represents a fully software-based, massively distributed public platform for edge computing research~\cite{cappos2018edgenet,senel2021edgenet1,senel2021edgenet2}.
It consists of a public, modified and customized Kubernetes~\cite{kubernetes} cluster.
The EdgeNet software is available for anyone to download;
individuals and organizations can quickly deploy self-hosted edge nodes which are then made available to anyone using the platform.
The end result is a flexible and scalable global edge cloud simplifying research and prototype development in edge and cloud computing.
A similar, closely-related project is KubeEdge~\cite{xiong2018extend}, a software framework for extending Kubernetes clusters from the cloud to the edge.

Finally, a number of other, small-scale approaches also exist.
In~\cite{gedawy2016cumulus}, the authors propose Cumulus, a prototype testbed for edge compute offloading in \gls{IOT} with the explicit goal of providing a generic and heterogeneous platform for edge computing research.
The authors of~\cite{rimal2018experimental} propose an experimental testbed for two-tiered edge architectures in the context of \gls{FIWI} edge computing.
This testbed is then employed to validate wireless access and compute resource scheduling strategies in these systems.
In~\cite{yamanaka2021design}, the authors present an experimental edge computing testbed capable of determining the offload location of a workload based on desired latency parameters.
\cite{diao2019scalable} propose yet another Kubernetes-based testbed based for edge task offloading based on heterogeneous hardware including low-power single-board computers such as Raspberry Pis and Jetson Nanos.
Finally,\ \cite{moorthy2022cloudraft} proposes \gls{CLOUDRAFT}, a cloud-based framework for mobile network experimentation, with a focus on simplifying the management of testbed resources.
The goal of this project is to integrate, coordinate, share, and improve upon existing testbeds through a common interface.
