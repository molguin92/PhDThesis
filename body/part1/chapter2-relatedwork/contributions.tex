\section{Summary of the key contributions of this thesis}\label{sec:summary_contributions}
\todo[inline]{Needs tweaking to match scope and related work.}
\todo[inline]{Missing discussion, presentation, of alternative approaches explicitly.}
\todo[inline]{Challenges and how we tackled them.}


This thesis presents three core contributions to the existing body of research in edge computing, as well as a number of secondary ones.
Firstly, we introduce a methodological approach to studying system responsiveness versus resource consumption trade-offs in edge-bound latency-sensitive applications such as \gls{WCA} and \gls{NCS}.
This approach is based on the emulation of the client-side workload, while maintaining the \emph{real} server-side process as well as network stack.
We validate this methodology by example, presenting case studies of its application in the context of \gls{WCA} and \glspl{NCS}, and introduce a software framework for the orchestration of the edge computing testbeds necessary for its implementation.

Secondly, we present a deeper exploration of this methodology for \gls{WCA} and introduce the first ever model for the end-to-end emulation of human timing behavior in \gls{WCA}.
This model is built from a thorough characterization of these behaviors, the data for which we obtain from a comprehensive human-subject study.

Finally, as an extension and application of our first and second contributions, we explore the implications of our methodology and human user model on the optimization potential of \gls{WCA} deployments.
In particular, we study the sampling and energy consumption behaviors of these systems with and without considering realistic human behavior.
We conclude that significant improvements can be achieved through the use of our human user model.

\section{Structure of this dissertation}

This dissertation is structured into two parts.
\cref{part:summary} presents a summary of the research, introducing the topic, discussing related work, and highlighting key contributions.
Next, in \cref{part:publications} we present the publications that form the core of this thesis.

\cref{part:summary} is further structured into four main chapters.
\cref{chap:introduction} serves as an introduction to the topic and outlines the key contributions of this work.
It sets the stage for the subsequent chapters by providing a high-level overview of the research problem and its significance.

In \cref{chap:relatedwork}, we present theoretical background and related work.
The chapter provides a comprehensive review of the existing literature, highlighting the current state-of-the-art and identifying any gaps in knowledge.
We discuss context for our research, and establish the foundation for the contributions presented in the dissertation.

\cref{chap:contributions} is the core of \cref{part:summary} and provides a detailed summary of the contributions made in this dissertation.
We discuss methods, tools, and findings that were developed and obtained throughout the course of our research.
We aim to demonstrate the significance of our work and how it contributes to the field.

Finally, \cref{chap:conclusions} concludes the dissertation and outlines avenues for future research.
We summarize the key findings and contributions of the thesis, and briefly discuss the implications of our findings and contributions for the field.
This chapter also identifies open questions and areas for further research, providing opportunities to build on the work presented in this thesis.
