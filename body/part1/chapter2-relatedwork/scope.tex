\section{Scope of this work}

This dissertation presents a methodology for the study of feedback-loop \gls{CPS} systems deployed on Edge Computing and studies its potential applications and implications.
This methodology is based on the emulation of the desired application workload on top of the actual Edge infrastructure, including backend computing hardware and network technology used for client-server communication.
We do not in any way claim that this methodology is the best approach to studying these systems, but rather that it is a valid approach able to produce accurate, representative results.

The scope of \gls{CPS} systems is vast and diverse, making it unfeasible to investigate all possible variations of these systems in this work. 
Instead, we opt to study a representative subset of future applications that could potentially leverage Edge Computing resources to provide innovative solutions, step-based task-assistance \gls{WCA} and \glspl{NCS}.
By considering these two application domains, we demonstrate the applicability of our methodology and showcase its efficacy.

Specifically, \cref{paper:olguinmunoz2018demoscaling,paper:olguinmunoz2019edgedroid,paper:olguinmunoz2022cleave,paper:olguinmunoz2022ainur} introduce and discuss this methodology and the necessary tools for its implementation.
\cref{paper:olguinmunoz2018demoscaling,paper:olguinmunoz2019edgedroid} present a high level overview of this methodological approach as applied to step-based task-assistance \gls{WCA}.
The \gls{WCA} system employed for these works corresponds to the LEGO assistant devised by \citeauthor{chen2015early}\cite{chen2015early}, using a task with \num{7} distinct steps.


\cref{} extends the discussion from \cref{paper:olguinmunoz2018demoscaling}, providing more details on the implementation of the necessary measurement framework and the tooling employed.
This work presents the first empirical results obtained with our methodological approach, providing thus an initial validation of its utility for research.
\cref{paper:olguinmunoz2019edgedroid} also discusses in detail the assumptions that were made about human behavior and reactions.
We assume in these works a human which is impervious to poor system performance, and suffers no annoyance, fatigue, frustration, nausea or other shortcomings of real human users.
The result is a model of a user who responds in a precisely reproducible and deterministic manner to the same system stimulus every time.

In \cref{paper:olguinmunoz2022cleave}, we extend our methodology to the study of control systems for \glspl{NCS} on Edge Computing.
These applications are similarly difficult to benchmark as \gls{WCA}, particularly at scale on multi-tenant systems, due to their client-side complexity and extreme sensitivity to latency.
To address these challenges, we introduce a tool, \acs{CLEAVE}, for the emulation and subsequent deployment of these systems on edge computing infrastructure.
This tool allows us both emulate the physical components of a relatively simple control system plant and deploy real algorithms for its control.
The middleware software abstracts away the network from the development of these workloads, enabling quick prototyping and deployment.
We then present use-case scenarios demonstrating the scalability and flexibility of their approach by deploying scenarios with a large number of loops without the need for domain-specific hardware.
This work showcases the applicability of the methodology beyond wearable cognitive assistance and highlights the potential of our approach for other edge computing applications.

\cref{paper:olguinmunoz2022ainur} presents a tangential contribution.
It introduces the software framework used in \cref{paper:olguinmunoz2023realistic,paper:olguinmunoz2022cleave} for the orchestration of the edge computing testbeds on which the developed tools were deployed.
This framework represents an ancillary contribution, crucial for the research presented in these works.
Without it, the experimental approach described in these works would not have been feasible.

Next, in \cref{paper:olguinmunoz2021impact,paper:olguinmunoz2023realistic} we delve deep into the implementation of our methodology for \gls{WCA}, as well as study its implications for optimization of these systems.
Although the above approach to human behavior in \gls{WCA} discussed represents a useful initial approximation, it is nonetheless not a realistic model of it.
In \cref{paper:olguinmunoz2021impact} we therefore take the first major step towards such a model, presenting a deep characterization of human behavior in these applications.
We develop this characterization through a human subject study with a cohort of \num{40} participants who were asked to interact with an instrumented \gls{WCA} application.
System responsiveness is altered in real-time during each execution of the task, and we record participants' reactions by measuring task- and system-related metrics, as well as biometrics from sensors placed on their bodies.
Participants were also asked to fill out two personality indicator questionnaires, allowing us to later correlate individual personality traits and measured reaction to changes in system responsiveness.

\cref{paper:olguinmunoz2023realistic} then concludes this line of work, building upon the insights and data obtained in \cref{paper:olguinmunoz2021impact} to develop the first ever data-driven model of human timings for \gls{WCA}.
The model is validated against previously obtained results, both through simulated, controlled executions and deployments on a real edge computing testbed.
This work also explores potential implications of this model for \gls{WCA} system optimization potential, particularly in the domains of energy consumption and sampling strategies.
