\section{Scope of this work}

This dissertation presents a methodology for the study of feedback-loop \gls{CPS} systems deployed on Edge Computing and studies its potential applications and implications.
This methodology is based on the emulation of the desired application workload on top of the actual Edge infrastructure, including backend computing hardware and network technology used for client-server communication.
We do not in any way claim that this methodology is the best approach to studying these systems, but rather that it is a valid approach able to produce accurate, representative results.

The scope of \gls{CPS} systems is vast and diverse, making it unfeasible to investigate all possible variations of these systems in this work. 
Instead, we opt to study a representative subset of future applications that could potentially leverage Edge Computing resources to provide innovative solutions: \gls{WCA} and \glspl{NCS}.
By considering these two application domains, we demonstrate the applicability of our methodology and showcase its efficacy.

Specifically, a subset of the research works included in this dissertation presents and studies the application of our methodology for \gls{WCA}.
To limit the scope of our research in this area, we constrain our work to \emph{step-based task-assistance} \gls{WCA}, and no other categories of \gls{WCA} are considered.
We focus on this category for a couple of reasons.
Step-based task-assistance \gls{WCA} were among the first categories of \gls{WCA} to be developed~\cite{chen2015early}, and thus represent a more mature application class compared to other \glspl{WCA}.
These applications also feature a linear mode of execution, simplifying emulation, and their step-based nature makes the tasks easier to shorten or extend for experimental purposes.
The specific application used for these studies corresponds to the LEGO assistant devised by \citeauthor{chen2015early}\cite{chen2015early}.

A separate subset of the research presented in this work discusses a preliminary foray into the applications of our methodology for \glspl{NCS}.
Given the vastness and complexity of this field, we limit our exploration to simple, small-scale, single-plant systems.
In particular, the system used as a representative example for the effectiveness of our approach for \glspl{NCS} corresponds to the \gls{2D} inverted pendulum, using simple \gls{PD} and \gls{PID} controllers.

Finally, this dissertation also briefly discusses Edge Computing testbed research and automation as an ancillary contribution to the methodology.
Our work in this domain is limited to small-scale testbeds with up to a few dozen nodes, leveraging \glspl{SDR} paired with general-purpose \gls{COTS} compute hardware.
We do not target large-scale testbeds due to their complexity, both in terms of hardware and software.