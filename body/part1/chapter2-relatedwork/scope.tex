\section{Scope of this dissertation}
\glsresetall%

\glspl{CPS} and \gls{XR} applications are emerging technologies with the potential to transform a plethora of industries.
However, both these categories of applications have stringent latency requirements which make their deployment on traditional Cloud infrastructure unfeasible.
Delays in communication and processing can significantly impact their performance and user experience, and potentially even cause physical harm.
Cloud Computing, with its centralized, physically distant datacenters, may not be able to meet these strict latency requirements.
This is where Edge Computing, a novel paradigm for distributed and mobile computing, offers a promising solution.
Edge Computing enables deployment of computing resources at the edge of the network, closer to end-users.
This shift in paradigm with respect to Cloud Computing provides drastically reduced latency, bandwidth reduction, as well as context-sensitivity and data locality.
These are key factors for real-time, context-sensitive, low-latency applications such as \gls{CPS} and \gls{XR}.

\section{Scope of this work}

This dissertation presents a methodology for the study of feedback-loop \gls{CPS} systems deployed on Edge Computing and studies its potential applications and implications.
This methodology is based on the emulation of the desired application workload on top of the actual Edge infrastructure, including backend computing hardware and network technology used for client-server communication.
We do not in any way claim that this methodology is the best approach to studying these systems, but rather that it is a valid approach able to produce accurate, representative results.

The scope of \gls{CPS} systems is vast and diverse, making it unfeasible to investigate all possible variations of these systems in this work. 
Instead, we opt to study a representative subset of future applications that could potentially leverage Edge Computing resources to provide innovative solutions: \gls{WCA} and \glspl{NCS}.
By considering these two application domains, we demonstrate the applicability of our methodology and showcase its efficacy.

Specifically, a subset of the research works included in this dissertation presents and studies the application of our methodology for \gls{WCA}.
To limit the scope of our research in this area, we constrain our work to \emph{step-based task-assistance} \gls{WCA}, and no other categories of \gls{WCA} are considered.
We focus on this category for a couple of reasons.
Step-based task-assistance \gls{WCA} were among the first categories of \gls{WCA} to be developed~\cite{chen2015early}, and thus represent a more mature application class compared to other \glspl{WCA}.
These applications also feature a linear mode of execution, simplifying emulation, and their step-based nature makes the tasks easier to shorten or extend for experimental purposes.
The specific application used for these studies corresponds to the LEGO assistant devised by \citeauthor{chen2015early}\cite{chen2015early}.

A separate subset of the research presented in this work discusses a preliminary foray into the applications of our methodology for \glspl{NCS}.
Given the vastness and complexity of this field, we limit our exploration to simple, small-scale, single-plant systems.
In particular, the system used as a representative example for the effectiveness of our approach for \glspl{NCS} corresponds to the \gls{2D} inverted pendulum, using simple \gls{PD} and \gls{PID} controllers.

Finally, this dissertation also briefly discusses Edge Computing testbed research and automation as an ancillary contribution to the methodology.
Our work in this domain is limited to small-scale testbeds with up to a few dozen nodes, leveraging \glspl{SDR} paired with general-purpose \gls{COTS} compute hardware.
We do not target large-scale testbeds due to their complexity, both in terms of hardware and software.

Before these systems can become widely deployed, however, their real-world performance must be thoroughly understood.
\glspl{CPS} often operate in safety-critical environments, and poor \gls{QoS} in \gls{XR} can easily lead to users experiencing discomfort while using the application.
Accurate evaluation, characterization, and optimization of the performance of these systems is therefore critical to ensure reliability, safety, and user satisfaction.
Additionally, accurate performance evaluation enables efficient resource utilization, and enhances system scalability, in turn leading to lower operation costs and a reduced barrier of entry.

Evaluating the performance of Edge environments involving such applications is challenging, due to complex behaviors emerging from the interaction between computation and network in these systems.
Existing literature has attempted to tackle the performance evaluation of Edge systems through analytical, simulation, and experimental approaches.
However, the nature of Edge Computing environments, which involves tight integration between network and compute, requires vast and often unfeasible interdisciplinary competence.
Existing works have been constrained to narrow their scope by prioritizing the comprehensive modeling of either compute or network, resulting in a potential trade-off that may neglect the other aspect.
As we will see in \cref{sec:state-of-the-art}, this has led to an incomplete understanding of the overall performance of Edge systems and applications in literature.

In this dissertation we introduce and study the applicability of an alternative methodology for Edge performance evaluation based on a practical, experimental approach.
We aim to complement literature by providing a method which bridges the gap in knowledge left by existing approaches.
Our work centers of Edge Computing environments involving latency-sensitive applications such as \gls{XR} and \glspl{CPS}, as these are broadly understood to represent ``killer'' use-cases for the Edge.
We also focus exclusively on Cloudlet Computing-like setups (see \cref{tab:scope}).
Although \gls{MEC} is one of the main enablers for \gls{WCA} applications, we consider such network deployments to be \gls{OOS} due to the complexity and financial investment required to deploy a mobile network.
Likewise, we do not target Fog Computing deployments due to this paradigm being a poor fit for latency-sensitive applications.
Thus, although the methodology and tools introduced are architecture-agnostic and should be directly applicable to other kinds of Edge Computing setups than the one used in this dissertation, further work may be needed to validate our results and conclusions in the context of \gls{MEC} and Fog Computing.

The performance evaluation of Edge-enabled \gls{XR} encompasses a wide range of factors, making the scope of research in this area extensive and complex.
The scope of performance evaluation research for \glspl{CPS} is at least equally extensive.
Attempting to cover both these scopes comprehensively in a single dissertation would be prohibitively challenging.
Therefore, in this work, we approach these challenges by focusing on representative use-cases for these application classes.

Specifically, we employ \emph{step-based task-assistance} \gls{WCA} as a representative use case for \gls{XR}, and \glspl{NCS} for \glspl{CPS}.
We focus on these categories for a couple of reasons.
In the case of \gls{WCA}, step-based task-assistance was among the first categories of \gls{WCA} to be developed~\cite{chen2015early}, and thus represents a more mature application class compared to other \glspl{WCA}.
These applications also feature a linear mode of execution, simplifying emulation, and their step-based nature makes the tasks easier to shorten or extend for experimental purposes.

For \glspl{NCS}, we limit our exploration in this dissertation to simple, small-scale, single-plant systems.
The system used corresponds to the \gls{2D} inverted pendulum, using simple \gls{PD} and \gls{PID} controllers.
We have chosen this system for its relative simplicity as well as its broad use in the field of automatic control as one of the fundamental examples of linear control.
Through these examples, we aim to gain insights into the performance evaluation challenges and opportunities offered by Edge Computing these applications and systems, and contribute to the understanding of the broader fields of \gls{XR} and \gls{CPS} performance evaluation in distributed computing environments.

\bigskip

In the following  \cref{sec:state-of-the-art}, we will review the state of the art in the context of performance evaluation of Edge Computing environments, including analytical, simulation, and experimental approaches.
We will evidence how existing approaches fail to fully capture the intricacies of performance evaluation of latency-sensitive, feedback-loop applications on Edge Computing, and make the case for the need for an alternative approach.
We will further discuss existing experimental approaches to \gls{MAR} and \gls{NCS} research, and discuss their limitations.

Finally, we conclude this chapter in \cref{sec:contributions}, where we will discuss in details the contributions of this dissertation.
This includes a discussion of the key sub-contributions of each included publication.