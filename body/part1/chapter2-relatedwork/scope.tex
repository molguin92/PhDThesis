\section{Scope of the thesis}
\glsresetall%

\todo[inline]{Write}

\todo[inline]{
    Emphasize that apps are standing for a new class of application class that the community is interested in developing.
    Why?
    Utility.
    Weave into intro and scope.

    Not just a random mashup.
    Relevance of the combination!

    Prime reason for experimental evaluation.
    Capture effects of compute, network, and design of the application.
    Hard to capture otherwise.

    Build argument in state of the art that there is no current satisfying way to address this.
    Potentially orders of magnitude off.

    In scope: interested in meaningful performance evaluation (i.e.\ capture all the complexity!) Make sure we note this.
    Show how apps and system are intimately related to each other and complex to evaluate.
    Where do complexities arise.
    Understanding system performance is key and complex.

    Scope: define the challenges.
}

\todo[inline]{
    Write about in terms of relevance and challenges.
    Motivate a bit. And then tell what we're not interested in.

    First 3 paragraphs in a conventional papers are usually scope. 
}

\begin{enumerate}
    \item Performance evaluation.
    \item \gls{CPS} and \gls{MAR}
    \item Edge Computing 
\end{enumerate}

\glsresetall%
\glspl{CPS} and \gls{XR} applications are emerging technologies with the potential to transform a plethora of industries.
However, both these categories of applications have stringent latency requirements which make their deployment on traditional Cloud infrastructure unfeasible.
Delays in communication and processing can significantly impact their performance and user experience, and potentially even cause physical harm.
Cloud Computing, with its centralized, physically distant datacenters, may not be able to meet these strict latency requirements.
This is where Edge Computing, a novel paradigm for distributed and mobile computing, offers a promising solution.
Edge Computing enables deployment of computing resources at the edge of the network, closer to end-users.
This shift in paradigm with respect to Cloud Computing provides drastically reduced latency, bandwidth reduction, as well as context-sensitivity and data locality.
These are key factors for real-time, context-sensitive, low-latency applications such as \gls{CPS} and \gls{XR}.

Before these systems can become widely deployed, however, their real-world performance must be thoroughly understood.
\glspl{CPS} often operate in safety-critical environments, and poor \gls{QoS} in \gls{XR} can easily lead to users experiencing discomfort while using the application.
Accurate evaluation, characterization, and optimization of the performance of these systems is therefore critical to ensure reliability, safety, and user satisfaction.
Additionally, accurate performance evaluation enables efficient resource utilization, and enhances system scalability, in turn leading to lower operation costs and a reduced barrier of entry.

Nevertheless, evaluating the performance of Edge environments involving such applications is challenging, due to complex behaviors emerging from the interaction between computation and network in these systems.
Existing literature has attempted to tackle the performance evaluation of Edge systems through both analytical and simulation approaches.
However, the nature of Edge Computing environments, which involves tight integration between network and compute, vast and often unfeasible interdisciplinary competence.
Existing works have been constrained to narrow their scope by prioritizing the comprehensive modeling of either compute or network, resulting in a potential trade-off that may neglect the other aspect.
This has resulted in an incomplete understanding of the overall performance of Edge systems and applications in literature.

In this dissertation we introduce and study the applicability of an alternative methodology for Edge performance evaluation based on a practical, experimental approach.
We aim to complement literature by providing a method which bridges the gap in knowledge left by existing approaches.
Or work centers of Edge Computing environments involving latency-sensitive applications such as \gls{XR} and \glspl{CPS}, as these are broadly understood to represent ``killer'' use-cases for the Edge.

However, the performance evaluation of Edge-enabled \gls{XR} encompasses a wide range of factors, making the scope of research in this area extensive and complex.
The scope of performance evaluation research for \glspl{CPS} is at least equally extensive.
Attempting to cover both these scopes comprehensively in a single dissertation would be prohibitively challenging.
Therefore, in this work, we approach these challenges by focusing on representative use-cases for these application classes.
These correspond to \gls{WCA} applications for \gls{XR}, and \glspl{NCS} for \glspl{CPS}.
Through these examples, we aim to gain insights into the performance evaluation challenges and opportunities offered by Edge Computing these applications and systems, and contribute to the understanding of the broader fields of \gls{XR} and \gls{CPS} performance evaluation in distributed computing environments.
