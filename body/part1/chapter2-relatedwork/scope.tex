\section{Scope of the thesis}
\glsresetall%

\todo[inline]{Write}

\todo[inline]{
    Emphasize that apps are standing for a new class of application class that the community is interested in developing.
    Why?
    Utility.
    Weave into intro and scope.

    Not just a random mashup.
    Relevance of the combination!

    Prime reason for experimental evaluation.
    Capture effects of compute, network, and design of the application.
    Hard to capture otherwise.

    Build argument in state of the art that there is no current satisfying way to address this.
    Potentially orders of magnitude off.

    In scope: interested in meaningful performance evaluation (i.e.\ capture all the complexity!) Make sure we note this.
    Show how apps and system are intimately related to each other and complex to evaluate.
    Where do complexities arise.
    Understanding system performance is key and complex.

    Scope: define the challenges.
}

This dissertation focuses on the performance evaluation of networked systems in the context of \glspl{CPS} and \gls{MAR} applications enabled by low-latency distributed computing paradigms.
Particular emphasis will be put on the role of Edge Computing as an enabling technology for these categories of systems and applications.

As discussed in \cref{sec:intro}, \glspl{CPS} and \gls{MAR} applications have both garnered the fascination of the research community due to their transformative potential and unique technical requirements.
\glspl{CPS} have the potential to transform entire industries through real-time monitoring and control and hitherto inviable optimization of physical processes.
\gls{MAR} applications are poised to transform the way we interact with the world around us, enabling novel forms of entertainment, education, training, and communication.
Nevertheless, both these classes of systems have stringent latency and context-awareness requirements which until recently had not been met by existing distributing computing paradigm.

Edge Computing plays therefore a crucial role in enabling these systems and applications.
Edge Computing enables low to ultra-low latencies while providing the necessary compute capabilities and context-sensitivity necessitated by \glspl{CPS} and \gls{MAR}.
However, Edge Computing represents a challenging area to benchmark, as the tight integration between compute and communication in these systems results in highly complex behaviors.
Unlike traditional 

\todo[inline]{WIP}


% Unlike traditional computing systems, edge computing involves a distributed network of devices that work together to perform complex tasks. This creates significant challenges for benchmarking, as it is difficult to isolate individual components and measure their performance in isolation. 


% Additionally, the performance of edge computing systems is highly dependent on the communication between devices, which can be impacted by a wide range of factors, including network topology, traffic patterns, and interference. As a result, benchmarking edge computing systems requires careful consideration of these factors, as well as the design and implementation of benchmarking tools that can accurately capture the complex interactions between compute and communication. Ultimately, benchmarking edge computing systems will be critical for ensuring that they meet the performance requirements of real-world applications and can deliver the benefits of low-latency, context-aware computing at scale.



% By processing data locally on edge devices, rather than sending it to centralized cloud servers, edge computing can significantly reduce the latency of CPS and MAR applications. This low-latency is particularly important in real-time applications that require rapid decision-making and response times, such as autonomous vehicles and remote surgery. In addition, edge computing provides context-awareness by enabling data to be processed in close proximity to the source, allowing for more efficient and effective use of data and reducing the need for bandwidth-intensive data transfers. Overall, the combination of low-latency and context-awareness provided by edge computing is essential for the successful deployment and operation of CPS and MAR applications in a wide range of industries and use cases.


% CPS have the ability to transform entire industries by optimizing physical processes, providing real-time monitoring and control, and making decisions based on data collected from various sources. In addition, the low-latency requirements of CPS enable rapid decision-making and response times, making them ideal for safety-critical applications. On the other hand, AR applications have the potential to transform how we interact with the world around us, enabling new forms of entertainment, education, and communication. The low-latency requirements of AR applications are also critical, as even a small delay in response time can lead to a subpar user experience. Overall, the study of CPS and AR applications presents exciting opportunities to explore cutting-edge technologies and their potential to transform various industries and aspects of our daily lives.


% The dissertation aims to investigate the challenges associated with the performance evaluation of networked systems and to propose novel methodologies and tools to assess the performance of such systems in real-world scenarios. The research will focus on the evaluation of cyber-physical systems and augmented reality applications that rely on low-latency distributed computing paradigms, such as fog computing and edge computing.

% The research will also explore the role of edge computing as an enabling technology for such applications, and will investigate the benefits and limitations of edge computing in terms of performance, scalability, and reliability. The dissertation will propose new approaches for optimizing the performance of edge computing-based systems, and will evaluate their effectiveness in real-world scenarios.

% Overall, the scope of this dissertation is to contribute to the body of knowledge on the performance evaluation of networked systems, with a focus on cyber-physical systems and augmented reality applications enabled by low-latency distributed computing paradigms, and the role of edge computing as an enabling technology.

\begin{enumerate}
    \item Performance evaluation.
    \item \gls{CPS} and \gls{MAR}
    \item Edge Computing 
\end{enumerate}