\section{Overview of our contributions}\label{sec:contributions}

\begin{figure}
    \centering
    \begin{forest}
        for tree={% style of tree
            align=center,
            grow=0,reversed, % tree direction
            parent anchor=east,
            child anchor=west, % edge anchors
            edge={line cap=round},
            outer sep=+1pt, % edge/node connection
            rounded corners,minimum width=15mm,minimum height=8mm, % node shape
            l sep=7mm % level distance
        }
        [{\scshape Contributions}
            [Methodology
                [{Use-case\\studies} [\acs{WCA}] [\acsp{NCS}]]
                [{Tooling} [{EdgeDroid\\(\acs{WCA})}] [{\acs{CLEAVE}\\(\acsp{NCS})}] [Ainur\\(testbed)]]
            ]
            [{Implications of\\Accuracy in\\the Emulation}
                [{Characterization of\\human behavior in \acs{WCA}}]
                [{Realistic human\\model for \acs{WCA}}]
                [Exploration of avenues\\for optimization of \acs{WCA}]
            ]
        ]
        \end{forest}
    \caption{Hierarchical overview of the main and secondary contributions of this dissertation.}\label{fig:contribs}
\end{figure}

This thesis presents two key contributions and a number of sub-contributions to literature, as illustrated in \cref{fig:contribs}.
Firstly, we introduce a methodology for the performance evaluation of latency-sensitive, feedback-loop-based applications enabled by Edge Computing such as \gls{MAR} and \glspl{CPS}.
This methodology is validated by example through two representative use-cases, and we provide tooling for its implementation.
Secondly, we study the implications improved client-side emulation accuracy has on the results produced with our methodology.
These implications are again explored by example.
We design and implement of a detailed model of human user behavior in \gls{WCA}, and then use it to study the potential for optimization of these systems.

\medskip

Our first contribution corresponds to a methodological approach to the study and performance evaluation of Edge-enabled latency-sensitive applications. 
This approach is based on the novel idea of emulating \emph{only} the end-user component of such applications while using the \emph{real} network and compute components of the system.
In other words, our methodology involves replacing the client side of the system with commercial off-the-shelf (COTS) general-purpose computing devices running realistic software imitations of the desired behaviors, while retaining the real network and compute hardware at the backend.
This entails modeling and emulating human inputs in \gls{MAR} applications and the physical components of \glspl{CPS}.

We study and validate our methodology by example, investigating its applicability in two representative use-cases for \gls{XR} applications and \glspl{CPS}, respectively: \acf{WCA} and \acfp{NCS}.
\cref{paper:olguinmunoz2018demoscaling,paper:olguinmunoz2019edgedroid} discuss initial explorative studies into the application of our methodology for \gls{WCA}.
These works present the first empirical results obtained with our methodological approach, providing thus an initial validation of its utility for research, and further discuss the necessary tooling and software frameworks required for the implementation of our methodology.
In particular, we introduce in these works a trace-based emulation tool which we call EdgeDroid.
It uses a trace of inputs recorded in optimal conditions, the replay of which is controlled at run-time by a simple model of human behavior based on a \gls{FSM}.

\Cref{paper:olguinmunoz2022cleave} extends our methodology to the study of control systems for \glspl{NCS} on Edge Computing.
These applications are similarly difficult to benchmark as \gls{WCA}, particularly at scale on multi-tenant systems, due to their client-side complexity and extreme sensitivity to latency.
To address these challenges, we introduce yet another tool, \gls{CLEAVE}, for the emulation and subsequent deployment of these systems on edge computing infrastructure.
\gls{CLEAVE} allows us both to emulate the physical components of a relatively simple control system plant and deploy real algorithms for its control.
The middleware software abstracts away the network from the development of these workloads, enabling quick prototyping and deployment.
We then present use-case scenarios demonstrating the scalability and flexibility of their approach by deploying scenarios with a large number of loops without the need for domain-specific hardware.

In \cref{paper:olguinmunoz2022ainur} we also briefly discuss Edge Computing testbed research and automation as an ancillary contribution to the methodology.
This paper formally introduces the Ainur framework for Edge Computing testbed automation and workload orchestration.
This framework is employed in \cref{paper:olguinmunoz2023realistic,paper:olguinmunoz2022cleave} for the orchestration of the Edge Computing testbeds on which the developed tools were deployed.
Ainur represents an ancillary contribution, crucial for the research presented in these works.
Without it, the experimental approach described in these works would not have been feasible.

We argue that our approach presents several advantages over existing methods for the study of trade-offs in latency-sensitive applications.
Our methodology allows for efficient and repeatable evaluation of Edge environments while avoiding many of the complications associated with experimental evaluation.
Emulating the workload component reduces complexity by moving it into the software domain, allowing for easier horizontal scaling through the use of \gls{COTS} general-purpose hardware such as \glspl{SBC}.
This also reduces the barrier of entry to  research, as \glspl{SBC} are for the most part cheap and easily accessible, and further preserves the realism of effects stemming from the hardware and network.
In particular, the methodology allows us to capture effects due to network factors such as contention, congestion control, and medium access, which are often of stochastic or chaotic natures and complex to capture in simulations.
The softwarized nature of the client-side emulation allows for straightforward automation through general-purpose programming and scripting languages.
Repeatability and replicability are also enhanced, as repeating a study becomes a matter of re-running the workload on the same testbed, and studies can be replicated simply by obtaining the same or equivalent software workload and deploying it on a comparable testbed.
These are complex tasks to accomplish in real-world approaches, particularly when dealing with humans.

Furthermore, the introduced approach encompasses the entire system, including both the application and the infrastructure. 
This holistic approach to performance evaluation of Edge environments is a notable contribution to the literature, as it provides a comprehensive view of the system's performance in realistic conditions.
This elegant methodology, in combination with the EdgeDroid, \gls{CLEAVE}, and Ainur tools, provides a contemporary approach to studying latency-sensitive applications on Edge Computing.

\medskip

The second key contribution of this dissertation relates to the exploration of the question of what degree of detail about the physical world must be captured by a client-side emulation of a latency-sensitive, feedback-loop application on Edge Computing.
We explore this interrogative through the elaboration of a detailed model of human user behavior in \gls{WCA}, which we then combine with mathematical framework to study the potential for optimization of these applications.

\cref{paper:olguinmunoz2021impact} begins this work by presenting a characterization of human behavior in \gls{WCA}, developed through a human subject study in which participants interact with an instrumented application.
System responsiveness was altered in real-time during each execution of the task, and we recorded participants' reactions by measuring task- and system-related metrics, as well as biometrics from sensors placed on their bodies.
Participants were also asked to fill out two personality indicator questionnaires, allowing us to later correlate individual personality traits and measured reaction to changes in system responsiveness.

Next, in \cref{paper:olguinmunoz2023realistic} we introduce the first ever data-driven model of human timings for \gls{WCA}, built upon the data and insights collected for \cref{paper:olguinmunoz2021impact}.
The model is validated against previously obtained results, both through simulated, controlled executions and deployments on a real edge computing testbed.
The paper goes on to explore potential implications of this model for \gls{WCA} system optimization potential, particularly in the domains of energy consumption and sampling strategies.

We conclude that realistic, dynamic human behavior in the client-side emulation enables superior optimization approaches that would otherwise be inviable to implement.
The findings from this research highlight that simply focusing on system-related metrics or performance optimizations may not be sufficient, as human behavior and reactions play a critical role in the overall performance of such applications.
By considering human behavior as an integral part of the emulation process, we demonstrate that it is possible to develop more realistic and effective optimization approaches that take into account the intricacies of human-computer interaction.
These findings have significant implications for the design and optimization of \gls{WCA} applications, particularly in the areas of energy consumption, sampling strategies, and application lifetime.

% \medskip

% This line of work has some limitations.
% First of all, we constrain our work to \emph{step-based task-assistance} \gls{WCA}, and no other categories of \gls{WCA} are considered.
% We focus on this category for a couple of reasons.
% Step-based task-assistance \gls{WCA} was among the first categories of \gls{WCA} to be developed~\cite{chen2015early}, and thus represents a more mature application class compared to other \glspl{WCA}.
% These applications also feature a linear mode of execution, simplifying emulation, and their step-based nature makes the tasks easier to shorten or extend for experimental purposes.

% However, restricting our research to step-based \gls{WCA} means that our results may not be directly applicable to other types of \gls{WCA} and \gls{AR} applications.
% In particular, our characterization of human behavior in \cref{paper:olguinmunoz2021impact} and both our models of human behavior --- the \gls{FSM}-based model introduced in \cref{paper:olguinmunoz2018demoscaling,paper:olguinmunoz2019edgedroid} and the data-driven model from \cref{paper:olguinmunoz2023realistic} --- assume linear modes of operation.
% Additionally, the timing characteristics of step-based \gls{WCA} are such that an instruction remains valid for as long as it takes the user to perform it.
% The conclusions from these works therefore may not apply to applications with different execution flows, or in which instructions have a deadline after which they may become stale or invalid.
% Examples of such applications include, for instance, the \emph{Ping-Pong} assistant introduced in~\cite{chen2015early}.

% The user models are further limited in their realism.
% In \cref{paper:olguinmunoz2018demoscaling,paper:olguinmunoz2019edgedroid} we assume a human which is impervious to poor system performance, and suffers no annoyance, fatigue, frustration, nausea or other shortcomings of real human users.
% The result is a model of a user who responds in a precisely reproducible and deterministic manner to the same system stimulus every time.
% These limitations emerge from a lack of data of human behavior in these applications, as at the moment of the writing of these works no comprehensive model of human behavior existed in the relevant literature.
% As a consequence, these works do not consider the effects of human error on \gls{WCA} application runtimes and footprints, and neither do they consider the feedback effects of system responsiveness on human task performance.
% We attempt to address some of these limitations in \cref{paper:olguinmunoz2021impact,paper:olguinmunoz2023realistic};
% the resulting model is able to adapt its timing behavior in response to delay in the system.
% However, we were unable to capture any instances of subjects making a mistake during data collection in \cref{paper:olguinmunoz2021impact}, and thus neither the characterization or subsequent model consider human error during task execution.
% Furthermore, participants for the study were recruited from a pool of undergraduate students at \gls{CMU} in the \gls{USA}, ranging in age from \numrange{18}{25} years old.
% This means that findings may not be generalizable to other demographic groups.

% Finally, to emphasize the possible effects of network congestion and latency on \gls{WCA}, in these works we exclusively employ \acs{IEEE} 802.11n \SI{2.4}{\giga\hertz} Wi-Fi networks in an Cloudlet Computing-like setup.
% Although \gls{MEC} is one of the main enablers for \gls{WCA} applications, we consider such network deployments to be \gls{OOS} due to the complexity and financial investment required to deploy a mobile network.
% Likewise, we do not target Fog Computing deployments due to this paradigm being a poor fit for latency-sensitive applications.
% Thus, although the methodology and tools introduced are architecture-agnostic and should be directly applicable to other kinds of Edge Computing setups than the one used in this dissertation, further work may be needed to validate our results and conclusions in the context of \gls{MEC} and Fog Computing.

% \bigskip



% Given the vastness and complexity of the field of \glspl{NCS}, we limit our exploration in this dissertation to simple, small-scale, single-plant systems.
% The system used corresponds to the \gls{2D} inverted pendulum, using simple \gls{PD} and \gls{PID} controllers.
% Although a representative example for the effectiveness of our approach for \glspl{NCS}, this means that our conclusions may not directly translate to larger scale or more realistic systems.
% Additionally, our tool, \gls{CLEAVE}, assumes a simple request-response design in the sensor-controller-actuator loop.
% This does not match well with more complex controllers such as \gls{MPC} controllers, in which each sample produces multiple outputs from the controller.
% \gls{CLEAVE} will thus have to be modified and extended to work with such setups.
% Our study into the application of our methodology for \glspl{NCS} also relies, once again, exclusively on Wi-Fi-enabled Cloudlet Computing.
% As such, the same limitations with respect to \gls{MEC} and Fog Computing apply as with our line of work in \gls{WCA}.

% \bigskip

% Our work in this domain is limited to small-scale testbeds with up to a few dozen nodes, leveraging \glspl{SDR} paired with general-purpose \gls{COTS} compute hardware.
% Medium and large testbeds such as \gls{COSMOS} and \gls{POWDER} are considered \gls{OOS} due to their cost and complexity, both in terms of hardware and software.
% In terms of networking technologies, Ainur is designed to be as agnostic as possible with respect to the network stack employed.
% This dissertation includes examples using Ethernet, Wi-Fi and 4G \gls{LTE}.
% Finally, Ainur does not currently fully integrate with modern Cloud and Edge infrastructure management and deployment technologies such as OpenStack~\cite{openstack}, although it does integrate with some core Cloud-native technologies such as Docker and Ansible.
% Work is currently ongoing to extend the framework to modern mobile network technologies such as 5G and beyond-5G, as well as to general-purpose, Cloud- and Edge-native infrastructure stacks.