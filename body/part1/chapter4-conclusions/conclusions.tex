In this thesis, we have made three significant contributions to the field of edge computing, addressing the challenge of studying latency-sensitive applications in a scalable, repeatable manner.

The overarching contribution of this dissertation is a new methodological approach for studying the trade-offs between system responsiveness and resource consumption in latency-sensitive applications at the edge.
Our approach is based on the emulation of the client-side workload while maintaining the actual server-side process and network stack.
This allows us to maintain a high degree of realism in studies, while at the same time affording significant potential for scalability and repeatability.
We validate this approach and present case studies of its application in the context of \gls{WCA} and \glspl{NCS}, and introduce a software framework for the orchestration of edge computing testbeds necessary for its implementation.

A follow-up contribution concerns a deeper exploration of this methodology for \gls{WCA} and introducing the first-ever model for the end-to-end emulation of human timing behavior in \gls{WCA}.
We gather data from a comprehensive human-subject study and use it to build a model that provides a more accurate representation of human behavior in \gls{WCA}.
This allows us to more accurately evaluate the advantages of our methodology when applied to these systems.

Finally, as an extension and application of our first two contributions, we investigate the implications of our methodology and human user model on the optimization potential of \gls{WCA} deployments.
In particular, we compare the sampling and energy consumption behaviors of these systems with and without considering realistic human behavior.
Through our methodology, we obtain reliable results indicating that the use of our human user model leads to significant improvements in system performance and energy consumption.

Our work demonstrates the flexibility and advantages of an emulated approach for the study of latency-sensitve applications on edge computing.
Further, it highlights the interplay between system responsiveness and resource consumption in the design and optimization of these applications.
We believe that the methodology and human user model presented in this thesis provide a valuable tool for researchers and practitioners in this field, and we hope that our contributions will inspire further research in this area.
