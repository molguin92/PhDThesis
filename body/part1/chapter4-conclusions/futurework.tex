% \section{Open questions \& future work}

% Notwithstanding the important contributions of this dissertation, a number of key questions remain open for future research in this field.

% The first concern is the applicability of the methodology to other systems and applications in the context of Edge Computing.
% This thesis explores the methodology's use for step-based \gls{WCA} in depth, and also performs a less thorough case study on its benefits for \gls{NCS} research.
% The question remains if the methodology is suitable for
% \begin{inlineenum}
%     \item other, non-\gls{WCA} \gls{XR} applications, such as more general mobile \gls{AR} and \gls{VR}
%     \item other more general categories of \glspl{CPS}
% \end{inlineenum}.
% The results obtained thus far indicate that the methodology is applicable to a wide range of interactive and \gls{CPS} applications in Edge Computing, but these results should be confirmed by domain-specific studies.

% In accordance with the broader questions mentioned above, this thesis also raises specific questions related to its individual contributions to the literature.
% The EdgeDroid tool, introduced in papers \cref{paper:olguinmunoz2018demoscaling,paper:olguinmunoz2019edgedroid,paper:olguinmunoz2023realistic}, has the potential for further development to support other forms of \gls{WCA} applications.
% Currently, the tool is only designed for a specific class of step-based \gls{WCA} with a linear task model.
% However, the tool, and by extension the methodology, could be easily adapted to similar applications.

% We identify the potential for expanding the dataset presented in \cref{paper:olguinmunoz2021impact} to include a more diverse range of participants.
% The current dataset is limited to young undergraduate students from a highly selective university in the \gls{USA}.
% Adding participants from different demographic groups would enhance the representativeness of the data and increase the generalizability of the findings.
% This would also provide an opportunity to extend the human timing model presented in \cref{paper:olguinmunoz2023realistic} to better reflect the behavior of a broader range of users.

% The optimization framework described in \cref{paper:olguinmunoz2023realistic} can be enhanced in two ways.
% Firstly, the framework assumes that human execution times follow a Rayleigh distribution, but in reality, they are often modeled with an \gls{exGaussian} distribution.
% Secondly, the framework can be made more practical by incorporating the optimization of scenarios with multiple clients and a distributed solution.

% There are numerous avenues for further investigation in the optimization and development of \gls{WCA}.
% While this thesis has explored the optimization of sampling and energy consumption, there are several other dimensions of these systems that can be examined.
% These include resource allocation, task scheduling, user identification and classification, as well as guaranteeing both \gls{QoE} and \gls{QoS}.
% Each of these areas presents opportunities for future work and could lead to significant advancements in the field of \gls{WCA}.

% \medskip
% Our research on applying the methodology to \glspl{NCS} presents promising potential for future work.
% While \cref{paper:olguinmunoz2022cleave} provides an initial exploration, further investigation through more realistic and thorough experimentation is necessary to validate the results obtained by the framework.
% To this end, we aim to conduct larger-scale experiments that utilize cutting-edge networking technologies, including 5G and future advancements such as 6G.
% By exploring the full range of networking technologies and scenarios, we can ensure that our methodology is robust and applicable to a wide range of \gls{NCS}.

% We also intend to extend the functionality of the \gls{CLEAVE} tool significantly, primarily to target a wider variety of \glspl{NCS} deployments.
% Currently, \gls{CLEAVE} includes a small repertoire of emulations and controllers.
% We are planning to extend this with the goal of creating an open library of \glspl{NCS} to share with the community.

% \medskip
% Finally, we see tremendous opportunities for testbed-driven research in Edge Computing.
% Our current efforts in this context aim towards the extension of our Ainur framework and its integration with \gls{CHI}~\cite{keahey2020lessons}.
% We believe Ainur's unique focus on fully-automated end-to-end wireless Edge Computing experimentation makes it valuable addition to the \gls{CHI} stack.
% At the same time, integrating with \gls{CHI} will provide a number of features and functionalities which will greatly expand Ainurs potential, such as resource reservation, multi-tenant testbed access, and automated bare-metal and \gls{VM}-host instantiation.

% The \gls{EXPECA} testbed is undergoing a major expansion and improvement.
% %The aim is to enhance its capabilities and hardware to meet the ever-increasing demands of the industry.
% The expansion consists of three main components;
% \begin{inlineenum}
%     \item a hardware upgrade
%     \item a small scale, private 5G deployment
%     \item a campus-wide 5G deployment
% \end{inlineenum}.
% Firstly, the \gls{EXPECA}s hardware is receiving a complete overhaul.
% We are replacing our initial Raspberry Pi-based prototype with a number of industry grade servers.
% Secondly, a small-scale private 5G deployment using industrial-grade equipment is currently being deployed.
% This will provide a controlled environment for testing and evaluating 5G technologies and applications.
% The use of industrial-grade equipment ensures that the results obtained from the tests are reliable and can be directly applied to real-world scenarios.
% Thirdly, the testbed will be extended to a campus-wide deployment at \gls{KTH} in Stockholm.
% This will provide an opportunity for a larger-scale deployment of 5G and \gls{MEC} technologies and applications, allowing for testing in real-world settings.
% The campus-wide deployment will also serve as a platform for researchers and industry professionals to collaborate and exchange ideas.
% The combined efforts of the small-scale private deployment and the campus-wide deployment at \gls{KTH} will significantly enhance the capabilities and knowledge of the \gls{EXPECA} testbed, positioning it as a leading platform for 5G and \gls{MEC} research and innovation.

% In conclusion, the present dissertation provides a comprehensive examination of the methodology for \glspl{NCS} and \gls{WCA}, but there are still many open questions that need to be addressed in future research.
% The EdgeDroid tool and the dataset have the potential to be extended, and the optimization framework employed in the thesis could be improved.
% We also plan to extend the functionality of the \gls{CLEAVE} tool and perform large-scale experimentation on novel networking technologies.
% There are also significant opportunities for testbed-driven research in Edge Computing through the \gls{EXPECA} testbed.
