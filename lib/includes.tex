%%%%%%%%%%%%%%%%%%%%%%%%%%%%%% Packages %%%%%%%%%%%%%%%%%%%%%%%%%%%%%%
%% The following are needed for generating the DiVA page(s)
\usepackage{scontents}              %% Needed to save lang, abstract, and keywords
\usepackage{pgffor}                 %% includes the foreach loop

%% Basic packages

%% Links
\usepackage{url}                %% Support for breaking URLs

%% Colorize
%\usepackage{color}
\PassOptionsToPackage{dvipsnames, svgnames}{xcolor}

%% Various useful packages
\usepackage[normalem]{ulem}
\usepackage{soul}
\usepackage{xspace}
\usepackage{braket}

% to support units and decimal aligned columns in tables
% the option loads the binary prefixes
\usepackage[binary-units=true, locale=US]{siunitx}

\usepackage{balance}
\usepackage{stmaryrd}
\usepackage{booktabs}
\usepackage{graphicx}	        %% Support for images
\usepackage{multirow}	        %% Support for multirow columns in tables
\usepackage{tabularx}		%% For simple table stretching
\usepackage{mathtools}
\usepackage{algorithm} 
\usepackage{algorithmic}  
\usepackage{amsmath}
\usepackage[linesnumbered,ruled,vlined,algo2e]{algorithm2e}
% can't use both algpseudocode and algorithmic packages
%\usepackage[noend]{algpseudocode}
%\usepackage{subfig}  %% cannot use both subcaption and subfig packages
\usepackage{optidef}
\usepackage{float}		%% Suppor for more flexible floating box positioning
\usepackage{pifont}

%% some additional useful packages
\usepackage{rotating}	    	%% For text rotating
\usepackage{array}		%% For table wrapping
\usepackage{mdwlist}            %% various list-related commands
\usepackage{setspace}           %% For fine-grained control over line spacing

\usepackage{enumitem}           %% to allow changes to the margins of descriptions


%% If you are going to include source code (or code snippets)
\usepackage{listings}		    %% For source code listing
%%\usepackage[cache=false]{minted} %% For source code highlighting
%%\usemintedstyle{borland}

\usepackage{bytefield}          %% For packet drawings
%%----------------------------------------------------------------------------
%%   pcap2tex stuff
%%----------------------------------------------------------------------------
\usepackage{tikz}
\usetikzlibrary{arrows,decorations.pathmorphing,backgrounds,fit,positioning,calc,shapes}
\usepackage{pgfmath}	% --math engine
%\usepackage{xcolor}


\newcommand\bmmax{2}
\usepackage{bm} % bold math

\setlength{\marginparwidth }{4cm} %% needs to be set before the todnotes package is loaded
\usepackage{todonotes}

%% If you are going to include source code (or code snippets)
\usepackage{listings}		%% For source code listing
%%\usepackage[cache=false]{minted} %% For source code highlighting
%%\usemintedstyle{borland}

\usepackage{dirtytalk}

\usepackage{hyperref}
\usepackage[all]{hypcap}	%% prevents an issue related to hyperref and caption linking
%% setup hyperref to use the darkblue color on links
\hypersetup{colorlinks,breaklinks,
            linkcolor=darkblue,urlcolor=darkblue,
            anchorcolor=darkblue,citecolor=darkblue}

\usepackage{notoccite} % do not number captions based on their appearance in the TOC

% to enable rotated figures
\usepackage{rotating}		%% For text rotating

% to allow changes to the margins of descriptions
\usepackage{enumitem}

% Footnotes
\usepackage{perpage}
\usepackage[perpage,para,symbol]{footmisc} %% use symbols to ``number'' footnotes and reset which symbol is used first on each page


%% Managing titles
% \usepackage[outermarks]{titlesec}
%%%%%%%%%%%%%%%%%%%%%%%%%%%%%%%%%%%%%%%%%%%%%%%%%%%%%%%%%%%%%%%%%%%%%%
%\captionsetup[subfloat]{listofformat=parens}

% to include PDF pages
%\usepackage{pdfpages}



\usepackage{csquotes} % Recommended by biblatex

% to provide a float barrier use:
\usepackage{placeins}

%\usepackage{filecontents}          % to be able to store and write to a file) specific contents

\usepackage{comment}  %% Provides a comment environment

\usepackage{kthpaper}
