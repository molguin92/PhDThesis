\section{Background of \gls{WCA}}\label{sec:background}

\begin{figure}
    \centering
    \begin{subfigure}{\columnwidth}
        \centering
        \includegraphics[width=\columnwidth]{publications/2023EdgeDroid2/figs/task}
        \caption{%
            Overview of a task in a \gls{WCA}, composed of a series of steps.
            Each steps starts with an instruction being provided to the user and ends with the instruction for the next step.
            The \gls{WCA} continuously samples the task state, automatically triggering transitions between steps as correct (or incorrect) states are recognized.
        }\label{fig:task}
    \end{subfigure}\\
    \begin{subfigure}{\columnwidth}
        \centering
        \includegraphics[width=\columnwidth]{publications/2023EdgeDroid2/figs/step_time}
        \caption{%
            Breakdown of a step into its timing components.
            The instruction for step \( M \) and \( M + 1 \) are provided to the user at \( t_0 \) and \( t_{n+1} \), respectively.
            \( t_k | k \in \{1, \ldots, n \} \) correspond to the \gls{WCA} sampling instants for step \( M \), and \( t_c \) marks the instant at which the user finishes performing the instruction.
        }\label{fig:step}
    \end{subfigure}
    \caption{Key concepts in \gls{WCA}}
\end{figure}

\glspl{WCA} applications represent a category of novel, context-sensitive and highly-interactive \gls{AR} applications.
In this work we focus on a particular category --- ``step-based'' \gls{WCA} --- that have as their goal the guiding of a user through a sequential task.
Examples of such applications are the LEGO and IKEA assistants~\cite{Chen2015LEGO,Chen2018application}, in which users are guided step-by-step through the process of assembling a LEGO model and an IKEA lamp, respectively.

Step-based \glspl{WCA} operate analogously to how \gls{GPS} navigation assistants guide users, by seamlessly and continuously monitoring the progress of the user and autonomously providing relevant instructions and feedback.
The application follows the progress of the task in ``realtime'' by repeatedly sampling the state of the physical system, most commonly through video frames.
Whenever the assistant detects that the user has correctly or incorrectly performed an instruction, it provides a new instruction to either advance the task or correct the detected mistake.
The application otherwise remains silent and out-of-the-way of the user; that is, samples which do not generate a new instruction (e.g.~because they captured an intermediate or unfinished state, or simply noise) are silently discarded.
Herein lies one of the key characteristics of these applications: the user only consciously interacts with the application whenever they finish an instruction, and thus these are the \emph{only} points in time at which they can notice changes in system responsiveness.

In order to discuss these applications with precision we provide some definitions relating to their operation.
First of all, a \emph{step} is formally understood as a specific action to be performed by the user, described by a single instruction, and a \emph{task} consists of a series of steps to be executed in sequence (see \cref{fig:task}).
A step begins when the corresponding instruction is provided to the user, and ends when the instruction for the next step is provided; we call the time interval between these two events the \emph{step duration}.

\glspl{WCA} employ sampling, most commonly of video feeds, to keep track of the state of the real world.
Take \( \{ t_0, t_1, \ldots, t_{n + 1} \} \) a series of discrete and sequential sampling instants at which the \gls{WCA} captures the state of the physical system, as illustrated in \cref{fig:step}.
\( t_0 \) corresponds to the instant at which the instruction for step \( M \) is provided and the first sample is taken, and \( t_n \) to the instant at which the final sample (i.e.\ which captures the final state of step \( M \)) is taken. 
\( t_{n + 1} \) is then the instant at which the result for sample \( t_n \) is returned and the instruction for step \( M + 1 \) is provided.
We define \( t_c \) as the point in time at which the user finishes performing the instruction for step \( M \), and the intervals \( t_c - t_0 \), \( t_n - t_c \), and \( t_{n + 1} - t_n \) as the \emph{execution time}, \emph{wait time}, and \emph{last sample \gls{RTT}}, respectively, of a step \( M \).
The sum of the latter two values (i.e.\ the interval \( t_{n + 1} - t_c \)) we call \emph{\gls{TTF}}, a metric which we will repeatedly refer to in this paper as it directly describes the responsiveness of a \gls{WCA}.

% It should be noted 
% it must follow that (see also~\cite{olguinmunoz:impact2021})\footnote{\( U(a, b) \) represents the continuous uniform distribution in the open interval \( (a, b) \).}
% % \todo[inline]{I think it's an assumption rather than a must. Specifically, it is not uniform if the general execution time distribution is Rayleigh/ExpGaussian -vishnu}
% \begin{align}\label{eq:tc}
%     t_c &\thicksim U(t_{n - 1}, t_n)
% \end{align}

% Current implementations, such as those in~\cite{Chen2015LEGO,Chen2018application}, employ \emph{greedy} sampling.
% In this scheme, a new sample is immediately taken as soon as an acknowledgement is received for the previous sample.
% This means that the interval between these sampling instants is not necessarily constant, as it is subject to fluctuations due to resource contention on both the network and compute side.
% It also has as a consequence that acknowledgements must be provided even for those samples which do not cause the generation of a new instruction --- these acknowledgements are simply used for flow control and do not generate user-noticeable feedback.

\subsection{The effects of changes in responsiveness on human behavior}\label{ssec:plos}

In~\cite{olguinmunoz:impact2021} we studied the effects of system responsiveness on human behavior in step-based \gls{WCA} in a controlled experiment.
We employed a modified and instrumented version of the above-mentioned, step-based LEGO \gls{WCA}~\cite{Chen2015LEGO}.
\num{40} subjects interacted with the assistant, performing a \num{169}-step task while we altered the responsiveness in realtime and captured key application and task performance metrics.
Additionally, we also employed questionnaires to evaluate personality traits of the participants, and correlated these results with the task performance metrics.

We used \emph{delay} and execution time as the variables for system responsiveness and human behavior, respectively.
\emph{Delay} corresponded to a temporarily fixed time duration for the processing interval of each frame.
That is, if during a series of steps delay was set to \( D \) seconds, the feedback for each frame was provided to the user \( D \) seconds after frame capture.
The correlation between this variable and execution time was then studied.
To compensate for the distribution of \( t_c \)\footnote{\( t_c \), and conversely the wait time \( \mathcal{W} \) of a step, can be assumed to be uniformly distributed in the interval \( [t_{n - 1}, t_n] \) without loss of generality.}~\cite{olguinmunoz:impact2021}, in the present work we adjust the nominal delay \( D \) by a factor \num{1.5}.
We will refer to this adjusted delay as \emph{\glspl{TTF}}.

% Delay, however, was merely an experimental tool, and if we consider
% \begin{inlineenum}
%     \item that \( t_c \), and conversely the wait time \( \mathcal{W} \) of a step, can be assumed to be uniformly distributed in the interval \( [t_{n - 1}, t_n] \) without loss of generality~\cite{olguinmunoz:impact2021}
%     \item that for a step subject to delay \( D \), \( t_k - t_{k - 1} = D \) for all \( k \in \{1, 2, \ldots, n\} \)
% \end{inlineenum},
% then it follows that the expected \gls{TTF} for steps subject to a delay \( D \) can be expressed as

% \begin{align}
%     \mathbb{E}\left[\text{\gls{TTF}}\right] &= \mathbb{E}\left[\mathcal{W}\right] + D
%     = \mathbb{E}\left[U(t_{n - 1}, t_n)\right] + D\nonumber\\
%     &= \mathbb{E}\left[U(0, D)\right] + D
%     = \frac{1}{2}D + D = \frac{3}{2}D\label{eq:ttf}
% \end{align}

% Ergo, we can directly re-parameterize the data from~\cite{olguinmunoz:impact2021} to use \gls{TTF} instead of delay by simply multiplying the delay value of each step by \num{1.5}.

Our findings can be summarized as follows:
\begin{enumerate}
    \item System slow-down induces \emph{additional} behavioral slow-down which scales with the decrease in system responsiveness.
    Compared to the unimpaired case, participants were on average \SI{12}{\percent} slower when subject to a mean \gls{TTF} of \SI{2.475}{\second}, and \num{26} percentage points slower at a mean \gls{TTF} of \SI{4.5}{\second}.

    \item\label{item:speedup} At lower \glspl{TTF}, humans get faster at performing steps as the task progresses,
    For sequences of \num{12} steps in an unimpaired application state, humans executed the final four steps on average \SI{36}{\percent} faster than they did the first four.
    However, this effect is dampened by reduced system responsiveness, and actually inverts at the highest levels of system impairment; humans actually become progressively slower the longer they spend in a degraded system state.

    \item\label{item:remain} The effects of system slow-down on human behavior remain for a while even after system responsiveness improves.
    These effects are noticeable for at least \num{4} steps after the return to a high-responsiveness state. 
    
    \item The above effects are modulated by measures of individual levels of personality characteristics and focus.
    % We will discuss this in detail below in \cref{ssec:moderationeffects}
\end{enumerate}

\subsubsection{Moderation of effects by individual characteristics}\label{ssec:moderationeffects}

We recorded variables related to well-known individual differences, encompassing both the \glspl{BFI}~\cite{oliver:bfi1999}, and the \glspl{ITQ}~\cite{witmer1998measuring}.
Out of the individual-difference variables, the most salient effect on performance corresponded to \emph{neuroticism}, a \gls{BFI} trait linked to low tolerance for stress and high emotional reactivity, and which has previously been linked to higher \emph{delay discounting} rates~\cite{hirsh2008delay}.
Delay discounting is the tendency to devalue rewards for which one must wait; high rates, indicative of waiting intolerance, have been associated with negative social and academic outcomes.

% Bobby suggested we remove this figure and just give the pearson coeff.
% \begin{figure}
%     \centering
%     \includegraphics[width=\columnwidth]{figs/new_model/correlation_neuro_exectime.png}
%     \caption{%
%         Correlation between neuroticism and the mean execution time of the last four steps in segments of \num{12} steps subject to the same \gls{TTF} in~\textcite{olguinmunoz:impact2021}.
%         Pearson correlation coefficient \( \rho = 0.418 \), 2-tailed \( p < 0.05 \).
%     }\label{fig:neuroexectimecorrel}
% \end{figure}

In this work, we will use a normalized scale to describe neuroticism, derived from the minimum and maximum obtainable values for this variable in the \gls{BFI}~\cite{oliver:bfi1999}.
\emph{Low} and \emph{high} neuroticism will refer to the \( [0.0, 0.5) \) and \( [0.5, 1.0] \) ranges respectively.

Linear regression showed a significant correlation between individual neuroticism scores and the execution time late in a series of high-delay steps, \( \rho = 0.418 \), \num{2}-tailed \( p < 0.05 \)
Neuroticism was further identified as a modulating factor for the pacing effects through a \gls{PCA}.
Out of the three identified components, which cumulatively accounted for \SI{73.13}{\percent} of the variance in the results, neuroticism was included in the first two.
The effect of neuroticism was observed across all \glspl{TTF} and impairment durations in the tasks.


\begin{figure}
    \centering
    \includegraphics[width=\columnwidth]{publications/2023EdgeDroid2/figs/new_model/mu_fits_exgaussian_slice0.png}
    \caption{%
        \( \mu \) parameter of \gls{exGaussian} distributions fitted to execution times of the first four steps of segments of steps subject to the same \gls{TTF} in~\cite{olguinmunoz:impact2021}.
        Distributions were fitted using \gls{MLE}.
    }\label{fig:muexgaussian}
\end{figure}

\begin{figure*}
    \centering
    \includegraphics[width=\textwidth]{publications/2023EdgeDroid2/figs/new_model/dist_fits_neuro.png}
    \caption{%
        Example \gls{exGaussian} fits on execution times from steps \numrange{4}{8} in a segment of steps at the maximum experimental \gls{TTF}.
        The effects of neuroticism are clearly visible in the tail and the mean of the distributions.
    }\label{fig:fitsneuro}
\end{figure*}

Furthermore, we found that execution times, when grouped by experimental variables such as neuroticism, \gls{TTF}, and continuous segments of steps subject to the same \gls{TTF}, were well-fit by an \gls{exGaussian} distribution, as verified using Kolmogorov-Smirnov goodness-of-fit tests~\cite{massey1951kolmogorov}.
When grouping by level of neuroticism, \gls{TTF}, and \emph{slice}\footnote{%
In~\cite{olguinmunoz:impact2021} the \emph{slice} to which a step belongs to refers to whether the step occurred in the first, second, or third four-step segment of a sequence of steps subject to the same \gls{TTF}.
}, the best fit statistic was \ensuremath{0.028} (\ensuremath{p = 0.999}).
This distribution has an ample body of research supporting its suitability for the modeling of the timing of human actions and reaction times~\cite{Rohrer1994analysis,Palmer2011shapes,Marmolejo2022generalised}.
We found that the effects of neuroticism on execution times were clearly identifiable in the fitted distributions, in particular in their means and tails.
\cref{fig:muexgaussian} shows an example of this modulating effect, illustrating the behavior of the mean (\( \mu \) parameter) of \gls{exGaussian} distributions fitted to the execution times of the first four steps of segments of steps subject to the same \gls{TTF}.
Finally, \cref{fig:fitsneuro} shows an example of the effects of neuroticism on the fitted \gls{exGaussian} distributions for a specific group of execution times.
Higher neuroticism directly translates into a higher mean and longer tail.
