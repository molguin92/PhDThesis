%Benchmarking human-in-the-loop applications is complex.
%This limits reproducibility as well as feasibility of performance evaluations.
%In this paper we present EdgeDroid, a benchmarking suite that addresses these challenges.
%Our core idea rests on recording traces of these applications \jjw{-> application traces} which are then replayed out \jjw{remove out} in a controlled fashion based on an underlying model of human behavior \jjw{-> emulating human behaviors}.
%The traces are then exposed to the original backend compute process of the respective human-in-the-loop application, generating realistic feedback.
%This allows for an automated system that greatly simplifies benchmarking large scale scenarios.
%Our results confirm the utility of EdgeDroid as a tool for system designers, application developers and researchers.

%\jjw{Alternative version:
Many emerging mobile applications, including augmented reality (AR) and wearable cognitive assistance (WCA), aim to provide seamless user interaction. 
However, the complexity of benchmarking these human-in-the-loop applications limits reproducibility and makes performance evaluation difficult. In this paper, we present EdgeDroid, a benchmarking suite designed to reproducibly evaluate these applications.

Our core idea rests on recording traces of user interaction, which are then replayed at benchmarking time in a controlled fashion based on an underlying model of human behavior. 
This allows for an automated system that greatly simplifies benchmarking large scale scenarios and stress testing the application.
Our results show the benefits of EdgeDroid as a tool for both system designers and application developers.
%}