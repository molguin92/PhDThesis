\section{Outline of paper}

Main research question:

\emph{Implications of the interdependence between human behavior and system responsiveness in WCA applications}

\begin{enumerate}
    \item Introduction
    \item Background and related work
        \begin{enumerate}
            \item What is WCA?
            \item Why is it interesting?
            \item Example applications and systems.
            \item Studying the impact of WCA on edge computing systems (why do we need a model for WCA?).
            \item Trade-offs: responsiveness (i.e. sampling speed) vs network load.
            \item Discuss how responsiveness relates to execution times and subsequently to system utilization.
            \item How to study these effects? Experimentally, with a model.
            \item Introduce model of WCA, refer back to PLOS paper. 
            % \item Impact of sampling schemes on system load.
            % \item Include examples from other systems (control theory).
            % \item Previous work on sampling in WCA.
            % \item Research on sampling is hampered by the lack of a model for WCA.
            % \item Previous approaches to understanding WCA behavior and modeling it.
        \end{enumerate}
    \item Model for WCA
        \begin{enumerate}
            \item Reiterate reasons for model: correlations between step execution time and system responsiveness
            \item Discuss base data from PLOS paper
                \begin{enumerate}
                    \item Timing
                    \item Frames
                \end{enumerate}
            \item Timing model
            \item Frame generation model
            \item Discuss effects of neuroticism (? for this we would need Bobby I think)
            \item Implementation in Python
        \end{enumerate}
    \item Validation of WCA Model
    \begin{enumerate}
        \item Reference model implementation (naive model)
        \item Comparison between naive, empirical, theoretical model using zero-wait sampling.
        \begin{enumerate}
            \item Scaling behavior (1, 5, and 10 clients?)
        \end{enumerate}
        \item Highlight behaviors captured by realistic models that are not present in naive model.
        \item Primarily, the spread in durations. Realistic models get faster in optimal conditions, but also react more strongly to lowered responsiveness.
    \end{enumerate}

    \item Studying implications of responsiveness-network load trade-offs using the model
    \begin{enumerate}
        \item Sampling strategies as a way to alter this trade-off.
        \item Present basic strategies: zero-wait, ideal, hold, regular.
        \begin{enumerate}
            \item Only realistic models, not naive.
            \item Present comparisons between basic strategies, without any additional load: total task duration, time-to-feedback breakdowns, total number of samples.
        \end{enumerate}
        \item Can we do better? Introduce and discuss advanced strategy: adaptive aperiodic approach.
        \begin{enumerate}
            \item Maybe fold in Vishnu's globecom paper here, with modifications to target responsiveness instead of power consumption.
            \item Discuss parameters, and how values were settled on.
            \item Compare adaptive aperiodic with 2 best basic strategies without additional load.
            \item Introduce additional load to showcase the adaptability of the adaptive aperiodic approach.
        \end{enumerate}
    \end{enumerate}
    \item Conclusion
    \begin{enumerate}
        \item Future work and open questions
        \item Effects of individual differences (neuroticism, age, etc)
        \item Extend to other WCAs
        \item Extend to other applications
    \end{enumerate}
\end{enumerate}
\newpage