\subsubsection{Cloud Instances}\label{sec:cloud}

Integration with \ac{AWS} is handled by the \mintinline{python}{CloudInstances} class.
The implementation of this class uses \verb|boto3| for interaction with the \ac{AWS}, and thus needs access to appropriate credentials\footnote{%
    \url{https://boto3.amazonaws.com/v1/documentation/api/latest/guide/credentials.html}%
}.


Each instance of this class is associated with a specific \ac{AWS} region, and manages the end-to-end lifecycle of an arbitrary number of \ac{AWS} \ac{EC2} instances on said region.
Specifically, \mintinline{python}{CloudInstances} handles
\begin{enumerate*}[itemjoin={{; }}, itemjoin*={{; and }}]
    \item configuring the necessary prerequisites for deploying the instances (security groups and private access keys)
    \item deploying the instances and waiting for them to successfully boot
    \item shutting down and terminating the instances
    \item cleaning up access keys and security groups
\end{enumerate*}.
Instances of this class can be used as context managers, automating this way the termination of instances and cleanup of resources when execution exits the \mintinline{python}{with} block.

By default, \mintinline{python}{CloudInstances} spawns instances of type \verb|t3.micro|, with at least one security group allowing inbound \ac{SSH} traffic and unrestricted traffic between instances.
The type of instances spawned can be modified at runtime, and additional security groups can be added at will.

The spawned \ac{EC2} instances are created from a preconfigured \ac{AMI}, the ID of which must be passed to the \mintinline{python}{CloudInstances} instance at runtime.
This \ac{AMI} can be either be one of the default \acp{AMI} provided by \acp{AWS} --- e.g.\ a plain Ubuntu or CentOS \ac{AMI} --- or a custom \ac{AMI} which has been preconfigured by the user.
Note that although the \mintinline{python}{CloudInstances} instance places no restrictions on the \ac{AMI} used for deploying (with the exception that it needs to be a Linux-based \ac{AMI}), components such as the VPN layer (see \cref{sec:vpn}) depending on these cloud instances might have requirements of their own.