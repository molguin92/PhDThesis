\subsubsection{Swarm Layer}\label{sec:swarm}

\begin{listing}[tb]
\caption{Example usage of the \mintinline{python}{DockerSwarm} class.}\label{lst:swarm:example}
\begin{minted}{python}
with DockerSwarm() as swarm:
    swarm.deploy_managers(
        hosts={
            cloudlet: {'location': 'edge'},
            aws01: {'location': 'cloud'}
        }
    )
    swarm.deploy_workers(
        hosts={
            client1: {},
            client2: {}
        },
        location='user'
    )
    swarm.pull_image('ubuntu', tag='20.04')
    swarm.deploy_workload(
        specification=workload_spec
    )
\end{minted}
\end{listing}
    

This layer handles the configuration and deployment of a Docker Swarm over the controlled hosts (including cloud instances) for the orchestration of containerized workloads.
It is implemented as a class \mintinline{python}{DockerSwarm} with appropriate methods for
\begin{inlineenum}
    \item connecting and configuring manager and worker nodes to the Swarm
    \item fetching required Docker container images to all configured nodes
    \item finally, deploying distributed workloads as stacks of Docker containers
\end{inlineenum}.
Additionally, the class defines appropriate \mintinline{python}{__enter__} and \mintinline{python}{__exit__} magic methods allowing it to be used as a context manager.

A simple example of the usage of this class and its basic \gls{API} can be observed in \cref{lst:swarm:example}, and in the following we will detail the major steps in operating this class:
\begin{description}[]
    \item[Instantiating a \mintinline{python}{DockerSwarm} object.]
    Obtaining an instance of a Swarm object is straightforward, as the constructor takes no parameters.
    \item[Adding Swarm nodes.]
    Swarm manager and worker nodes are added through the \mintinline{python}{deploy_managers()} and \mintinline{python}{deploy_workers()} methods respectively.
    These methods take a single mandatory parameter \mintinline{python}{hosts}, corresponding a mapping from \mintinline{python}{AinurHost} instances to dictionaries of node variables in the form \mintinline[breaklines, breakafter={_}]{python}{{'label_name': 'label_value'}}.
    Additional keyword arguments are parsed as default values for labels which are assigned to all nodes.

    Note that the actual underlying Docker Swarm cluster is created on-demand at the time of the first call to \mintinline{python}{deploy_managers()}.
    This means that at least one call to \mintinline{python}{deploy_managers()} \emph{must} be made before any call to any other method of the Swarm.
    \item[Fetching required container images.]
    \mintinline{python}{DockerSwarm} provides a \mintinline{python}{pull_images()} method which takes a mandatory repository parameter, as well as an optional tag identifier, and triggers a download of the specified container image on all nodes in the Swarm.
    This method supports all repository and tag formats supported by the \verb|docker-py| library, and defaults to fetching from Docker Hub\footnote{\url{https://hub.docker.com/}} when the given repository name string does not include remote registry information.
    \item[Workload Deployment.]
    Finally the Swarm object also provides a method \mintinline[breaklines, breakafter={_}]{python}{deploy_workload()}, which takes a \emph{workload definition} and deploys the workload on the Swarm.
    Details on the workload definition and running workloads are given in \cref{sec:workload} 
\end{description}
