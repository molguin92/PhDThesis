\subsubsection{\gls{IP} Layer Connectivity}\label{sec:layer3}

Layer 3 connectivity, that is, \gls{IP} addressing and routing, is handled by two main components in Ainur: \mintinline{python}{LANLayer} and \mintinline{python}{VPNCloudMesh}.

\mintinline{python}{LANLayer} handles local network connectivity.
It can be used as a context manager for automatic network teardown, as well as an iterator over the hosts it manages.
This class exposes two main \gls{API} methods, \mintinline{python}{self.add_hosts()} and \mintinline{python}{self.tear_down()}:
\begin{description}[]
    \item[\mintinline{python}{LANLayer.add_hosts(self, layer2)}]
    This method takes a \mintinline{python}{PhysicalLayer} object instance and configures \gls{IP} layer connectivity on the hosts within.
    This includes assigning \gls{IP} addresses to interfaces, configuring default and static routes, as well as authenticating and connecting to the desired access point in the case of Wi-Fi.
    The underlying implementation uses Ansible to write Netplan YAML configuration files on the target hosts and subsequently apply the network configuration.

    \item[\mintinline{python}{LANLayer.tear_down(self)}]
    Performs a clean teardown of \gls{IP} layer connectivity of the hosts managed by the instance, removing the created Netplan configuration files and reverting host network configuration to its initial state.
    When \mintinline{python}{LANLayer} is used as a context manager, this method is automatically called the moment execution leaves the \mintinline{python}{with} block.
\end{description}

\mintinline{python}{VPNCloudMesh}\label{sec:vpn} handles \gls{VPN} connectivity to running cloud instances.
It makes some strong assumptions about the presence of two preconfigured \gls{VPN} gateways on the testbed (some of these have been mentioned above, but we will repeat them here for clarity):

\begin{itemize}
    \item Two VPNCloud gateways, one for control connections and one for workload data streams, are configured and accessible on the control network.
    \item One of these gateways is additionally connected to the workload data network through a separate interface.
    \item These gateways are both listening on separate \gls{UDP} ports on a single public IP address.
    In other words, both gateways access the public internet through a single \gls{NAT} gateway, on which appropriate \gls{UDP} port forwarding rules are configured to allow \gls{VPN} traffic back into the control network.
    \item The VPNCloud connections configured on the the gateways share the same pre-shared key.
    \item The VPNCloud links belong to mutually \emph{disjoint} \gls{IP}v4 address ranges.
    % \item The gateways are assigned the first host address in the available address space of their respective \gls{VPN} links.
    % For example, if the gateway \gls{VPN} link address space is \texttt{10.0.0.0/24}, the gateway has address \texttt{10.0.0.1} on the \gls{VPN} link.
    \item Devices on the control network have been preconfigured with appropriate \gls{IP} routes to reach the control gateway and its associated \gls{VPN} link.
    \item Finally, cloud instances intended to be used with this class have VPNCloud installed on them.
\end{itemize}

After spawning cloud instances with an instance of the \mintinline{python}{CloudInstances} class described in \cref{sec:cloud}, the \mintinline{python}{VPNCloudMesh} class can be used to connect said cloud instances to the control and workload data networks of the testbed.
Instances of this class can further be used as context managers for automatic cleanup of stale links, and implement an iterator interface to access connected hosts.

The \gls{API} of this class exposes the below relevant methods:

\begin{description}[]
    \item[The \mintinline{python}{VPNCloudMesh} constructor]
    
    The constructor takes the following arguments:

    \begin{description}[font=\normalfont\ttfamily\underline]
        \item[gateway\_ip] An \texttt{IPv4Address} object corresponding to the shared public \gls{IP}v4 address of the \gls{VPN} gateways (i.e.\ the public \gls{IP} address of the \gls{NAT} gateway.)
         
        \item[vpn\_psk]
        The pre-shared key used to authenticate and encrypt traffic on the \gls{VPN} links.

        \item[ansible\_ctx~\normalfont{and} ansible\_quiet]
        Correspond to an Ansible context to use for the execution of this instances' methods (see above), and a boolean flag to suppress verbose debug output, respectively.

        \item[gw\_mgmt\_ip~\normalfont{and} gw\_wkld\_ip]
        Full \gls{IP} addresses of the \gls{VPN} links for the control and workload data networks, respectively, including subnet masks.
        These values are passed as \mintinline[breakafter=4]{python}{IPv4Interface} objects, and are used to set up correct routing tables to the cloud instances.

        \item[mgmt\_local\_net~\normalfont{and} wkld\_local\_net]
        The base addresses of the local control and workload data networks, respectively, as \mintinline[breakafter=4]{python}{IPv4Network}.
        Used to configure routing back from the cloud.

        \item[mgmt\_port~\normalfont{and} wkld\_port]
        \gls{UDP} ports on the \gls{NAT} gateway forwarded to the control and workload gateways, respectively.
    \end{description}
    
    \item[\mintinline{python}{VPNCloudMesh.connect_cloud(self, ...)}]
    
    The full signature of this method is \mintinline[breakafter=.,breakafter=_]{python}{VPNCloudMesh.connect_cloud(self, cloud_layer, host_configs)}.
    The \texttt{cloud\_layer} argument corresponds to a \mintinline{python}{CloudInstances} object instance, and the \texttt{host\_configs} to a collection of host configuration dataclasses.
    This method matches the host configurations to cloud instances in the \mintinline{python}{CloudInstances} object and connects them to to the local control and workload data networks through the two aforementioned separate \gls{VPN} links.
    The underlying implementation uses a combination of the \texttt{boto3} library and Ansible to configure appropriate security groups on the instances and subsequently initialize and connect the \gls{VPN} links.

    \item[\mintinline{python}{VPNCloudMesh.tear_down(self)}]
    This method tears down the \gls{VPN} links, disconnecting the remote instances from the gateways and removing any created routes to the cloud from local devices.
    When an instance of this class is used as a context manager, this method is automatically called at the end of the \mintinline{python}{with} block, ensuring automatic cleanup of resources.
    
\end{description}
